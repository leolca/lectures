\subsection{Revisão Huffman}

\begin{frame}[allowframebreaks]
  \frametitle{Codificação de Huffman}
  \begin{itemize}
  \item Codificação de Huffman é uma codificação de símbolo. Um símbolo é codificado por vez.
  \item A codificação de Huffman é ótima para comprimentos inteiros.
  \item De forma geral, cada símbolo do alfabeto da fonte deve ser codificado por um número inteiro de bits 
	formando uma palavra.
  \item Para distribuições que não são D-ádicas, podemos estar utilizando um bit extra por símbolo, na média.
  \item Em alguns casos, precisaremo de longos blocos para ter benefício, ou seja, codificar em blocos.
  \item Será necessário computar e armazenar $p(x_{1:n})$.
  \item Para um alfabeto de tamanho $\vert \mathcal{X} \vert$ será necessário armazenar/calcular 
	$\vert \mathcal{X} \vert^n$ probabilidades.
  \item Existem problemas na estimação de $p(x_{1:n})$ (esparsividade e latência). 
	Será necessário utilizar técnicas de \textit{smoothing}.
  \item Para Huffman teremos
	\begin{equation}
	H(X) \leq L_{\text{Huffman}} \leq H(X)+1
	\end{equation}
  \item Para Huffman em blocos
	\begin{equation}
        H(X) \leq L_{\text{Huffman em blocos}} \leq H(X)+1
        \end{equation}
  \item Se $H(X_{1:n})$ é pequeno, um bit extra é significante.
  \end{itemize}
\end{frame}

\begin{frame}[allowframebreaks]
  \frametitle{As Probabilidade se Alteram}
  \begin{itemize}
  \item Processos sequenciais reais não são estacionários. Muitas vezes é razoável supor que os processos
	são `localmente estacionários', ou seja, para uma determinada janela do tempo poderemos considerar
	que o processo é governado por uma distribuição $p(x)$ fixa.
  \item Huffman assume que $p(x)$ é fixo. Se a distribuição muda para $p'(x)$ em outro instante do tempo
	sabemos que agora a codificação será sub-ótima e o preço pago por utilizar a distribuição errada
	é dado por $D(p'(x) \mid \mid p(x))$ bits por símbolo, onde $p'(x)$ será a distribuição correta
	para esta nova janela no tempo.
  \item Possível solução para o problema:
	\begin{itemize}
	\item Recalcular Huffman utilizando a nova distribuição (ou estimação desta) para cada período.
		Esta solução é ineficiente pois implica em retransmitir o codebook a cada intervalo,
		sendo para tanto necessário aguardar um período para estimar a nova distribuição e então 
		recalcular Huffman.
	\item Podemos fazer um método de Huffman adaptativo.
	\end{itemize} 
  \end{itemize}
\end{frame}


\subsection{Huffman Adaptativo}
\begin{frame}[allowframebreaks]
  \frametitle{Huffman Adaptativo}
  \begin{itemize}
  \item A cada símbolo codificado e transmitido iremos recalcular as frequências de ocorrências 
	(e assim as probabilidades estimadas) e inserir símbolos na árvore de Huffman ou alterar
	a árvore, caso necessário.
  \item Codificador e Decodificador trabalham realizando as mesmas alterações (\textit{lockstep}).
  \item É necessário um símbolo extra NYT para indicar que teremos uma nova folha na árvore, um
	símbolo que não foi transmitido previamente.
  \end{itemize}

  \begin{example}
  Vamos considerar a sequência $aabcdad$ no alfabeto $\{a,b,\ldots\}$ a ser codificada utilizando
  Huffman Adaptativo.
  
  Tabele com código fixo dos símbolos, sem utilizar a codificação de Huffman:

	\begin{tabular}{ll}
	símbolo & código \\ \hline
	a	& 000	\\
	b	& 001	\\
	c	& 010	\\
	d	& 011 	\\
	$\vdots$& $\vdots$
	\end{tabular}

  \examplebreak
                \begin{figure}[h!]
                \centering
                \includegraphics[width=0.5\textwidth]{images/ahuff3.pdf}
                \label{fig:ahuff3}
                \end{figure}
  \examplebreak
                \begin{figure}[h!]
                \centering
                \includegraphics[width=0.5\textwidth]{images/ahuff4.pdf}
                \label{fig:ahuff4}
                \end{figure}

  \examplebreak
                \begin{figure}[h!]
                \centering
                \includegraphics[width=0.5\textwidth]{images/ahuff5.pdf}
                \label{fig:ahuff5}
                \end{figure}

  \examplebreak
                \begin{figure}[h!]
                \centering
                \includegraphics[width=0.5\textwidth]{images/ahuff6.pdf}
                \label{fig:ahuff6}
                \end{figure}

  \examplebreak
                \begin{figure}[h!]
                \centering
                \includegraphics[width=0.5\textwidth]{images/ahuff7.pdf}
                \label{fig:ahuff7}
                \end{figure}

  \examplebreak
                \begin{figure}[h!]
                \centering
                \includegraphics[width=0.5\textwidth]{images/ahuff8.pdf}
                \label{fig:ahuff8}
                \end{figure}

  \examplebreak
                \begin{figure}[h!]
                \centering
                \includegraphics[width=0.5\textwidth]{images/ahuff9.pdf}
                \label{fig:ahuff9}
                \end{figure}

  \examplebreak
                \begin{figure}[h!]
                \centering
                \includegraphics[width=0.5\textwidth]{images/ahuff10.pdf}
                \label{fig:ahuff10}
                \end{figure}

  \examplebreak
                \begin{figure}[h!]
                \centering
                \includegraphics[width=0.5\textwidth]{images/ahuff11.pdf}
                \label{fig:ahuff11}
                \end{figure}

  \examplebreak
                \begin{figure}[h!]
                \centering
                \includegraphics[width=0.5\textwidth]{images/ahuff12.pdf}
                \label{fig:ahuff12}
                \end{figure}

  \examplebreak
                \begin{figure}[h!]
                \centering
                \includegraphics[width=0.5\textwidth]{images/ahuff13.pdf}
                \label{fig:ahuff13}
                \end{figure}

  \examplebreak
                \begin{figure}[h!]
                \centering
                \includegraphics[width=0.5\textwidth]{images/ahuff14.pdf}
                \label{fig:ahuff14}
                \end{figure}

  \examplebreak
                \begin{figure}[h!]
                \centering
                \includegraphics[width=0.5\textwidth]{images/ahuff15.pdf}
                \label{fig:ahuff15}
                \end{figure}

  \examplebreak
                \begin{figure}[h!]
                \centering
                \includegraphics[width=0.5\textwidth]{images/ahuff16.pdf}
                \label{fig:ahuff16}
                \end{figure}

  \examplebreak
                \begin{figure}[h!]
                \centering
                \includegraphics[width=0.5\textwidth]{images/ahuff17.pdf}
                \label{fig:ahuff17}
                \end{figure}

  \examplebreak
                \begin{figure}[h!]
                \centering
                \includegraphics[width=0.5\textwidth]{images/ahuff18.pdf}
                \label{fig:ahuff18}
                \end{figure}

  \examplebreak
                \begin{figure}[h!]
                \centering
                \includegraphics[width=0.5\textwidth]{images/ahuff19.pdf}
                \label{fig:ahuff19}
                \end{figure}

  \examplebreak
                \begin{figure}[h!]
                \centering
                \includegraphics[width=0.5\textwidth]{images/ahuff20.pdf}
                \label{fig:ahuff20}
                \end{figure}

  \examplebreak
                \begin{figure}[h!]
                \centering
                \includegraphics[width=0.5\textwidth]{images/ahuff21.pdf}
                \label{fig:ahuff21}
                \end{figure}

  \examplebreak
                \begin{figure}[h!]
                \centering
                \includegraphics[width=0.5\textwidth]{images/ahuff22.pdf}
                \label{fig:ahuff22}
                \end{figure}

  \examplebreak
                \begin{figure}[h!]
                \centering
                \includegraphics[width=0.5\textwidth]{images/ahuff23.pdf}
                \label{fig:ahuff23}
                \end{figure}

  \examplebreak
                \begin{figure}[h!]
                \centering
                \includegraphics[width=0.5\textwidth]{images/ahuff24.pdf}
                \label{fig:ahuff24}
                \end{figure}

  \examplebreak
                \begin{figure}[h!]
                \centering
                \includegraphics[width=0.5\textwidth]{images/ahuff25.pdf}
                \label{fig:ahuff25}
                \end{figure}

  \examplebreak
                \begin{figure}[h!]
                \centering
                \includegraphics[width=0.5\textwidth]{images/ahuff26.pdf}
                \label{fig:ahuff26}
                \end{figure}

  \examplebreak
                \begin{figure}[h!]
                \centering
                \includegraphics[width=0.5\textwidth]{images/ahuff27.pdf}
                \label{fig:ahuff27}
                \end{figure}

  \examplebreak
                \begin{figure}[h!]
                \centering
                \includegraphics[width=0.5\textwidth]{images/ahuff28.pdf}
                \label{fig:ahuff28}
                \end{figure}

  \end{example}
\end{frame}

