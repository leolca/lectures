\subsection{Métrica e Distância}
% Cover and Thomas Cap02 Q15 (solution) / Q09 (livro)

\begin{questions}
\question{
Uma função $f:S \times S \rightarrow \mathbb{R}$ é uma métrica em $S$ se, para todo $x,y,z \in S$, as seguintes condições são satisfeitas:
\begin{itemize}
\item $f(x,y) \geq 0$ (não negatividade)
\item $f(x,y) = f(y,x)$ (simetria)
\item $f(x,y) = 0$ se e somente se $x=y$ (identidade dos indiscerníveis)
\item $f(x,y) + f(y,z) \geq f(x,z)$ (desigualdade triangular).
\end{itemize}

Uma função $f:S \times S \rightarrow \mathbb{R}$ é uma distância em $S$ se, para todo $x,y \in S$, as seguintes condições são satisfeitas:
\begin{itemize}
\item $f(x,y) \geq 0$ (não negatividade)
\item $f(x,y) = f(y,x)$ (simetria)
\item $f(x,x) = 0$.
\end{itemize}

\begin{parts}
\part
Mostre através de um exemplo que a divergência de Kullback-Leibler (entropia relativa) não é uma distância e sequer uma métrica.

\part
Mostre que $\rho(X,Y) = H(X|Y) + H(Y|X)$ possui as propriedades de uma métrica. Note que $\rho(X,Y)$ é o número de bits
necessário para $X$ e $Y$ comunicarem seus valores um para o outro.

\part
Verifique que $\rho(X,Y)$ também pode ser expresso como
\begin{eqnarray}
\rho(X,Y) &=& H(X) + H(Y) - 2 I(X;Y) \nonumber \\
          &=& H(X,Y) - I(X;Y) \nonumber \\
          &=& 2 H(X,Y) - H(X) - H(Y)
\end{eqnarray}

\end{parts}

}


\begin{solution}
\begin{parts}
\part
A divergência de Kullback-Leibler não é simétrica. Exemplo: considere $p = (\frac{1}{2}, \frac{1}{2})$ e 
$q = (\frac{1}{4}, \frac{3}{4})$. Teremos assim:

  \begin{equation}
  D(p||q) = \frac{1}{2} \log \frac{1/2}{3/4} + \frac{1}{2} \log \frac{1/2}{1/4} = 1 - \frac{1}{2} \log 3 = 0.2075 \text{bits,}
  \end{equation}
  \begin{equation}
  D(q||p) = \frac{3}{4} \log \frac{3/4}{1/2} + \frac{1}{4} \log \frac{1/4}{1/2} = \frac{3}{4} \log 3 - 1 = 0.1887 \text{bits.}
  \end{equation}

  Podemos observar que neste exemplo temos $D(p||q) \neq D(q||p)$, ou seja, a divergência de Kullback-Leibler 
  não é simétrica.

\part

 \begin{itemize}
 \item (não-negatividade)
        Como $H(X|Y) \geq 0$ e $H(Y|X) \geq 0$, teremos que $\rho(X,Y) \geq 0$.
 \item (simetria)
        Segue pela definição: $\rho(X,Y) = H(X|Y) + H(Y|X) = \rho(Y,X)$.
 \item (identidade dos indiscerníveis)
        Se $X=Y$, então $H(X|Y) = H(Y|X) = 0$ e assim $\rho(X,Y) = 0$.
        Por outro lado, se $\rho(X,Y) = 0$, então $H(X|Y) + H(Y|X) = 0$, mas
        $H(X|Y) \geq 0$ e $H(Y|X) \geq 0$, logo devemos ter $H(X|Y) = H(Y|X) = 0$,
        então $X=Y$.
 \item (desigualdade triangular)
        \begin{eqnarray}
                && \text{condicionar não aumenta a entropia} \nonumber \\
        H(X|Y) + H(Y|Z) &\geq& H(X|Y,Z) + H(Y|Z) \nonumber \\
                && \text{regra da cadeia} \nonumber \\
                &=& H(X,Y|Z) \nonumber \nonumber \\
                && \text{regra da cadeia} \nonumber \\
                &=& H(X|Z) + \underbrace{H(Y|X,Z)}_{\geq 0} \nonumber \\
                &\geq& H(X|Z)
        \end{eqnarray}
        da mesma forma podemos mostrar que $H(Z|Y) + H(Y|X) \geq H(Z|X)$.
        Teremos assim
        \begin{eqnarray}
        H(X|Y) + H(Y|Z) + H(Z|Y) + H(Y|X) &\geq& H(X|Z) + H(Z|X) \nonumber \\
        \rho(X,Y) + \rho(Y,Z) &\geq& \rho(X,Z)
        \end{eqnarray}
 \end{itemize}


\part

\begin{eqnarray}
\rho(X,Y) &=& \underbrace{H(X|Y)}_{H(X) - I(X;Y)} + \underbrace{H(Y|X)}_{H(Y) - I(X;Y)} \nonumber \\
        &=& H(X) + H(Y) - 2 I(X;Y) \nonumber \\
        && \text{utilizando $H(X,Y) = H(X) + H(Y) - I(X;Y)$} \nonumber \\
        &=& H(X,Y) - I(X;Y) \nonumber \\
        && \text{utilizando $I(X;Y) = H(X) + H(Y) - H(X,Y)$} \nonumber \\
        &=& 2 H(X,Y) - H(X) - H(Y)
\end{eqnarray}
  

\end{parts}
\end{solution}
\end{questions}
