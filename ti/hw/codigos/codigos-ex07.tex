\subsection{Comprimento esperado (Shannon e Huffman)}

\begin{questions}
\question{
    Suponha uma fonte que produza os símbolos do alfabeto
    $\mathcal{X}=\{A,B,C,D,E\}$ com distribuição
    $p = (\nicefrac{2}{5}, \nicefrac{1}{5}, \nicefrac{1}{5}, \nicefrac{1}{10}, \nicefrac{1}{10})$.
    Usando $\log 5 = 2,32$, calcule $H(X)$, $L_s$ e $L_H$ 
    (entropia da fonte, comprimento esperado do código de Shannon e
    comprimento esperado do código Huffman, respectivamente).
}

\begin{solution}
    A entropia é dada por
    \begin{eqnarray}
    H(X) &=& - \nicefrac{2}{5} \log \nicefrac{2}{5} - 2\times \nicefrac{1}{5} \log \nicefrac{1}{5} - 2 \times \nicefrac{1}{10} \log \nicefrac{1}{10} \\
                &=& \nicefrac{2}{5} (\log 5 - 1) + \nicefrac{2}{5} \log 5 + \nicefrac{1}{5} (\log 5 + 1) \\
                        &=& \log 5 (\nicefrac{2}{5} + \nicefrac{2}{5} + \nicefrac{1}{5}) - \nicefrac{2}{5} + \nicefrac{1}{5} = \log 5 - \nicefrac{1}{5} = 2,32 - 0,2 = 2,12 .
    \end{eqnarray}
    
    Os comprimentos do código de Shannon são dados por $\lceil - \log p(x) \rceil$.
    Teremos então os seguintes comprimentos:
    \begin{equation}
    l(A) = \lceil - \log \nicefrac{2}{5} \rceil = \lceil \log 5 - 1 \rceil = \lceil 2,32 - 1 \rceil = \lceil 1,32 \rceil = 2
    \end{equation}
    \begin{equation}
    l(B) = l(C) = \lceil - \log \nicefrac{1}{5} \rceil = \lceil \log 5 \rceil = \lceil 2,32  \rceil = 3
    \end{equation}
    \begin{equation}
    l(D) = l(E) = \lceil - \log \nicefrac{1}{10} \rceil = \lceil \log 10 \rceil = \lceil \log 5 + 1 \rceil = \lceil 2,32 + 1 \rceil = \lceil 3,32 \rceil = 4
    \end{equation}
    assim o comprimento esperado do código de Shannon será
    \begin{equation}
    L_S = \nicefrac{2}{5} \times 2 + 2 \times \nicefrac{1}{5} \times 3 + 2 \times \nicefrac{1}{10} \times 4 = \nicefrac{28}{10} = 2,8
    \end{equation}

    
    O código de Huffman será $(A,(B,(C,(D,E))))$ e assim os comprimentos do
    código de Huffman serão $l(A) = 1$, $l(B) = 2$, $l(C) = 3$ e $l(D)=l(E)=4$.
    O comprimento esperado deste código será
    \begin{equation}
    L_H = \nicefrac{2}{5} \times 1 + \nicefrac{1}{5} \times 2 + \nicefrac{1}{5} \times 3 + 2 \times \nicefrac{1}{10} \times 4 = \nicefrac{22}{10} = 2,2
    \end{equation}

\end{solution}
\end{questions}

