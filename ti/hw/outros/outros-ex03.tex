\subsection{Comportamento no limite}
% harvard / HW2_ES250_sol_a

\begin{questions}
\question{
Sejam $X_1,X_2,\ldots$ tirados de forma i.i.d. de acordo com a seguinte distribuição:
\begin{equation}
        X_i = 
        \begin{cases} 
        1, \quad 1/2, \\
        2, \quad 1/4, \\
        3, \quad 1/4.
        \end{cases}
\end{equation}
Encontre o comportamento limite para o produto
\begin{equation}
        (X_1 X_2 \ldots X_n)^{1/n}
\end{equation}

}

\begin{solution}
Vamos definir
\begin{equation}
P_n = (X_1 X_2 \ldots X_n )^{\frac{1}{n}} = \left( \prod_{i=1}^{n} X_i \right)^{\frac{1}{n}} , 
\end{equation}
assim teremos
\begin{equation}
\log P_n = \frac{1}{n} \sum_{i=1}^{n} \log X_i \rightarrow \E [\log X]  ,
\end{equation}
pela lei forte dos grandes números. Assim, poderemos concluir que
$P_n \rightarrow 2^{\E [\log X]}$.

Dada os valores que $X_i$ assume e sua distribuição, podemos calcular
\begin{equation}
\E [\log X] = \frac{1}{2} \log 1 + \frac{1}{4} \log 2 + \frac{1}{4} \log 3 = \frac{1}{4} \log 6  .
\end{equation}
Desta forma, teremos $P_n \rightarrow 2^{\frac{1}{4} \log 6} = 6^{1/4} = 1.565$.

\end{solution}
\end{questions}
