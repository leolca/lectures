\subsection{Média dos valores quadráticos}
% ~/ee/ufsj/2017_02/ti/provas/prova01-resolucao.tex

\begin{questions}
\question{
Mostre que a média dos valores quadráticos excede (ou é igual)
o quadrado da média dos valores. Em qual situação haverá igualdade?
Justifique todos os passos.
}

\begin{solution}
Iremos utilizar a desigualdade de Jensen, utilizando que $f(x) = x^2$ é uma função convexa.
Suponha que tenhamos $N$ valores, $x_i$, $i=1,\ldots,N$.
  \begin{proof}
    \begin{eqnarray}
    \frac{1}{N} \sum_{i=1}^{N} x_i^2 &=& \sum_x \frac{1}{N} f(x) = \sum_x p(x) f(x) \nonumber \\
        &\geq& f \left( \sum_x p(x) x \right) = \nonumber \\
        &=& \left( \sum_{i=1}^{N} \frac{1}{N} x_i \right)^2 = \left( \frac{1}{N} \sum_{i=1}^{N} x_i \right)^2 
    \end{eqnarray}
  \end{proof}

Haverá igualdade quando $x_i = 1$, $\forall i$, pois $x=1$ é o único que satisfaz $x^2 = x$.
Neste caso teremos $\frac{1}{N} \sum_{i=1}^{N} 1^2 = 1$ e $\left( \frac{1}{N} \sum_{i=1}^{N} 1 \right)^2 = 1$.
\end{solution}
\end{questions}
