\subsection{Soma módulo}
% bilmes Homework 5 - q2  - similar Harvard HW5_ES250_sol.pdf q4

\begin{questions}
\question{
Considere o canal discreto sem memória $Y = (X + Z) \mod 11$, onde
\begin{equation}
        Z = \begin{pmatrix}1, & 2, & 3 \\ 1/3, & 1/3, & 1/3 \end{pmatrix}
\end{equation}
e $X \in \{0,1,\ldots,10\}$. Considere $Z$ independente de $X$. 

\begin{parts}
\part 
Calcule a capacidade do canal.

\part 
Qual é $p^\ast(x)$ que maximiza?

\end{parts}
}

\begin{solution}
\begin{parts}
\part 

Sabemos que $I(X;Y) = H(Y) - H(Y|X)$. E a capacidade de canal é $C = \max_{p(x)} I(X;Y)$.
Temos que 
\begin{equation}
H(Y|X) = H(Z|X) = H(Z) = \log 3 .
\end{equation}
Assim,
\begin{eqnarray}
C &=& \max_{p(x)} I(X;Y) \nonumber \\
        &=& \max_{p(x)} H(Y) - \log 3 \nonumber \\
        &=& \log 11 - \log 3 = \log \frac{11}{3}
\end{eqnarray}
onde utilizamos que a informação mútua será máxima quando $Y$ possuir 
distribuição uniforme e, consequentemente, $X$ também terá distribuição uniforme,
uma vez que $Y = (X + Z) \mod 11$ e $Z$ possui distribuição uniforme.


\part 

Conforme dado acima, a distribuição uniforme é aquela que maximiza a informação mútua entra $X$ e $Y$,
$p^\ast(x) = \left( \frac{1}{11}, \frac{1}{11}, \ldots, \frac{1}{11} \right)$.


\end{parts}
\end{solution}
\end{questions}
