\subsection{O estatístico e o canal}
% cover thomas q1
\begin{questions}
\question{
É dado um canal de comunicação com as probabilidades de transição
$p(y\vert x)$ e a capacidade de canal $C = \max_{p(x)} I(X;Y)$.
Um estatístico resolveu ajudar e propôs processar a saída através de uma
determinada função $\tilde{Y} = g(Y)$. Ele alega que isto irá 
melhorar a capacidade do canal estritamente.

\begin{parts}
\part 

Mostre que ele está errado.

\part 

Sob quais condições ele não irá estritamente diminuir a capacidade?


\end{parts}
 
}

\begin{solution}
\begin{parts}
\part 

Ao adotar $\tilde{Y} = g(Y)$, teremos $X \rightarrow Y \rightarrow \tilde{Y}$, 
formando uma cadeia de Markov. Poderemos assim aplicar a desigualdade de processamento
de dados
\begin{equation}
I(X;Y) \geq I(X;\tilde{Y}) .
\end{equation}

Seja $\tilde{p}(x)$ a distribuição em $x$ que maximiza $I(X;\tilde{Y})$, fornecendo
a capacidade do canal $\tilde{C}$ entre $X$ e $\tilde{Y}$, teremos:
\begin{eqnarray}
C &=& \max_{p(x)} I(X;Y) \\
        &\geq& I(X;Y)_{p(x)=\tilde{p}(x)} \text{(onde $\tilde{p}$ é a dist. que max. $I(X;\tilde{Y})$)} \\
        &\geq& I(X;\tilde{Y})_{p(x)=\tilde{p}(x)} \text{ (utilizando a desigualdade de proc. de dados)}\\
        &=& \max_{p(x)} I(X;\tilde{Y}) = \tilde{C}.
\end{eqnarray}
Logo, qualquer processamento subsequente sobre $Y$ que for realizado não irá aumentar a capacidade do canal.


\part 

Para que não ocorra uma diminuição da capacidade de canal, deveremos ter igualdades
na sequência de desigualdades acima. Para isso será necessário ter igualdade na
desigualdade de processamento de dados, ou seja, $I(X;Y) = I(X;\tilde{Y})$.
Isto ocorrerá quando, dada a distribuição $\tilde{p}(x)$ que maximiza $I(X;\tilde{Y})$,
tivermos a seguinte cadeia de Markov $X \rightarrow \tilde{Y} \rightarrow Y$.


\end{parts}
\end{solution}
\end{questions}
