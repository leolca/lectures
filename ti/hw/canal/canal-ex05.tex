\subsection{Máquina de escrever}
% harvard HW5_ES250_sol.pdf - q2
\begin{questions}
\question{
Considere uma máquina de escrever com 26 teclas.
\begin{parts}
\part 
Se ao pressionar uma tecla temos como resultado a impressão da letra associada, 
qual é a capacidade $C$ em bits?

\part 
Agora suponha que ao pressionar uma tecla podemos ter como resultado a impressão 
da letra associada ou da letra vizinha (com igual probabilidade). 
Desta forma $A \rightarrow A \textmd{ ou } B$, $\ldots$, $Z \rightarrow Z \textmd{ ou } A$. 
Qual é a capacidade agora?

\part 
Qual é o código de maior taxa, com blocos de comprimento unitário, que você consegue 
encontrar que alcança probabilidade zero de erro para o canal do item anterior?

\end{parts}
}

\begin{solution}
\begin{parts}
\part 
\begin{eqnarray}
C &=& \max_{p(x)} I(X;Y) \nonumber \\
        &=& \max_{p(x)} H(X) - \underbrace{H(X|Y)}_{=0} = \max_{p(x)} H(X) = \log 26 = 1 + \log 13
\end{eqnarray}


\part 
\begin{eqnarray}
C &=& \max_{p(x)} I(X;Y) \nonumber \\
        &=& \max_{p(x)} H(X) - \underbrace{H(X|Y)}_{=H(\frac{1}{2})} = \max_{p(x)} H(X) - 1= \log 26 - 1 = \log 13
\end{eqnarray}

\part 
Podemos utilizar um código de blocos de comprimento unitário utilizando as letras alternadas,
isto é, A, C, E, G, etc. Desta forma não haverá confusão e a taxa deste código será
\begin{equation}
R = \frac{\log (\text{num. de palavras})}{\text{tamanho do bloco}} = \frac{\log 13}{1} = \log 13 .
\end{equation}
Desta forma, iremos alcançar a capacidade do canal.


\end{parts}
\end{solution}
\end{questions}
