\subsection{Capacidade dada matriz 2}

\begin{questions}
\question{
Determine a capacidade do canal descrito pelo matriz de transição abaixo.
\begin{equation}
  P_{y \vert x} = 
        \begin{bmatrix} 
        p       & 0     & 1-p   & 0     \\
        0       & q     & 0     & 1-q   \\
        1-p     & 0     & p     & 0     \\
        0       & 1-q   & 0     & q
        \end{bmatrix}
\end{equation}

}

\begin{solution}
Note que este canal de comunicação é equivalente ao canal dado abaixo
onde trocamos os papeis dos símbolos 2 e 3:
\begin{equation}
  P_{y \vert x} = 
\begin{blockarray}{ccccc}
 1 & 3 & 2 & 4 \\
\begin{block}{(cccc)c}
  p   & 1-p & 0   & 0   & 1 \\
  1-p & p   & 0   & 0   & 3 \\
  0   & 0   & q   & 1-q & 2 \\
  0   & 0   & 1-q & q   & 4 \\
\end{block}
\end{blockarray}
\end{equation}

Este canal, como já vimos anteriormente, é constituído pela soma de dois canais
binários simétricos, com capacidade $C_1 = 1 - H(p)$ e $C_2 = 1 - H(q)$.
Combinados, estes canais terão capacidade
\begin{eqnarray}
C &=& \log \sum_{i=1}^{2} \left( 2^{C_i} \right)  \nonumber \\
        &=& \log \left( 2^{C_1} + 2^{C_2} \right) \nonumber \\
        &=& \log \left( 2^{1 - H(p)} + 2^{1 - H(q)} \right)
\end{eqnarray}

\end{solution}
\end{questions}
