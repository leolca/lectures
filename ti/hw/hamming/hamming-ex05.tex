\subsection{Hamming Unicode}

\begin{questions}
\question{
No Unicode, a letra `a' possui o ponto de representação \texttt{U+0061}. Este, por sua vez,
é codificado no UTF-8 como a sequência de 8 bits a seguir: $01100001$.
\begin{parts}
\part Utilizando o código de Hamming (7,4,3) codifique esta sequência de bits que representa o caractere `a'
para serem transmitidos através de um canal ruidoso.
\part Suponha que um ruído aditivo esteja presente no canal, acarretando a troca dos bits nas posições 3 e 12
(considerando a sequência completa de bits formada após a codificação de Hamming).
Utilize a decodificação do código de Hamming para detectar erro na transmissão e corrigi-lo. Mostre como.
\part ($A \rightarrow B \rightarrow C$) 
        Suponha que o receptor B trabalha com uma extremidade de bit (\textit{bit endianness})\footnote{
                Ordenação dos bits, convenção adotada para identificar a posição dos bits em um número binário.
                LSB 0: o bit menos significativo está no primeira posição. MSB 0: o bit mais significativo
                está na primeira posição.
                } 
        diferente daquela do transmissor A.
O receptor B recebe a sequência de bits, enviados pelo transmissor A, através de um canal com ruído, 
e aplica a verificação e correção de erros do código de Hamming. 
Em seguida transmite os dados, através de um canal sem ruído, para um terceiro C que utiliza a mesma extremidade que
o transmissor original A da mensagem. Como esta última transmissão foi através de um canal sem ruído, o
último receptor C não precisa se preocupar com detecção e correção de erros.
A correção de erros realizada pelo receptor intermediário B danificará os dados, impossibilitando que o 
receptor final C decodifique corretamente a mensagem? Justifique. 

\end{parts}
}

\begin{solution}
\begin{parts}
\part 
Os dados $0110 | 0001$ darão origem a duas palavras do código de Hamming (7,4,3): 
$0110011$ e $0001111$ (usando as matrizes vistas em aula) ou $1100110$ e $1101001$ (usando as matrizes criadas pelo algoritmo).
Estas são dadas pela multiplicação $x=Gp$, onde $p$ são os bits de dados, $x$ a palavra de hamming e $G$ a matriz geradora.

\part 
O sinal recebido será dado por $y = x + r$, onde $r$ é o ruído aditivo.

Para realizar a verificação e correção de erros devemos calcular a síndrome: $s = Hy$.
Se a síndrome for nula, não há erro, caso contrário, devemos encontrar a coluna de $H$ igual à síndrome.
Para corrigir o erro bastará trocar o bit na posição correspondente àquela onde encontramos $s$ em $H$.

Como no exercício em questão temos apenas um erro em cada palavra de Hamming, ao proceder a correção este erro
será corrigido.

\part 
O receptor C receberá e decodificará a mensagem sem erro pois B irá corrigir os erros existentes.
Como B trabalha com a extremidade de bit inversa, ele irá ter a matriz de verificação de paridade invertida, 
que será utilizada com o vetor de bits invertido. 
Isto equivale a trocar a ordem das equações de verificação de paridade, o que não muda em nada o funcionamento
do sistema. Desta forma, tudo funcionará normalmente, porém na ordem inversa.
Para C isso será transparente e não fará diferença.

Se B não usar a matriz invertida em relação a A, ou seja, B estaria utilizando uma matriz invertida em relação
ao código de Hamming, teremos erro na decodificação, exceto nos casos em que a palavra transmitida seja um palíndromo.

\end{parts}
\end{solution}
\end{questions}
