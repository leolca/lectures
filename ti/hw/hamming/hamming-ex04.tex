\subsection{Hamming}
\begin{questions}
\question{
Um código de Hamming utiliza palavras de comprimento igual a 7 bits.
Suponha que esta mensagem, codificada com este código, será enviada
através de um canal binário simétrico com probabilidade de troca de bit 
$p=0.1$. Faça o que se pede abaixo.
\begin{parts}
\part Determine a taxa deste código.
\part Calcule a probabilidade de comunicação sem erro.
\part Calcule a probabilidade de detecção de erro, dado que a comunicação teve erro.
\part Faça exemplos de codificação, decodificação e decodificação com correção de erro,
para ilustrar todos os possíveis casos de operação.
\end{parts}
}

\begin{solution}
\begin{parts}
\part 
Trata-se do código de Hamming (7,4,3), utiliza 4 bits de dados e 3 bits de paridade.

\part 
A taxa do código é $4/7 \approx 0,571$.

\part 
A comunicação será sem erro se a transmissão for sem erro, ou se for possível
corrigir o erro. Para o código de Hamming (7,4,3) só é possível detectar e corrigir 
no máximo um único bit trocado na sequência.
Para que a transmissão seja sem erro, nenhum bit pode ser trocado. Isto 
ocorrerá com probabilidade $(1-p)^7 \approx 0,47830$.
Para que ocorra apenas um erro, a probabilidade será dada por $7 \times p(1-p)^6 \approx 0,37201$.
Desta forma o percentual de comunicação sem erro será $(1+6p) (1-p)^6 \approx 0,47830 + 0,37201 = 0,85031$.

\part 
Ocorrerá algum erro com probabilidade $1-0,47830 = 0,52170$. O erro será detectado se
for de um número ímpar de bits trocados. Desta forma, a probabilidade dos casos em que 
é detectado erro é dada por
\begin{equation}
{7 \choose 1} p (1-p)^6 + {7 \choose 3} p^3 (1-p)^4 + {7 \choose 5} p^5 (1-p)^2 + {7 \choose 7} p^7  \approx 0,39513 ,
\end{equation}
logo, dado que ouve erro na comunicação, a probabilidade de corrigir o erro é de $0,39513/0,52170 = 0,75739$

\part 
Ver exemplos apresentados em aula.
 
\end{parts}
\end{solution}
\end{questions}
