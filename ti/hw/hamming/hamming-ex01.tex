\subsection{Código de Hamming (7, 4, 3)}

\begin{questions}
\question{
Código de Hamming $(7,4,3)$
        \begin{equation} \label{eq-mod2-h743a}
        x_4 = (x_1 + x_2 + x_3) \mod 2 
        \end{equation}
        \begin{equation} \label{eq-mod2-h743b}
        x_5 = (x_0 + x_2 + x_3) \mod 2
        \end{equation}
        \begin{equation} \label{eq-mod2-h743c}
        x_6 = (x_0 + x_1 + x_3) \mod 2
        \end{equation}

        \begin{equation}
        H = 
        \begin{pmatrix}
        0 & 1 & 1 & 1 & 1 & 0 & 0 \\
        1 & 0 & 1 & 1 & 0 & 1 & 0 \\
        1 & 1 & 0 & 1 & 0 & 0 & 1 
        \end{pmatrix}.
        \end{equation}


Considere os números $a$ e $b$, o penúltimo e o último algarismos da sua matrícula, respectivamente.
O dado a ser enviado é a representação binária com 4 bits de $((10 a + b) \mod 16)$
(exemplo: se a matrícula fosse 114350047, teríamos $a=4$ e $b=7$, assim $47 \mod 16 = 15$ e 
o dado em binário seria $1111$).

\begin{parts}
\part
Escreva sequência $x$ que representa o código que será enviado utilizando a
codificação de Hamming (7,4,3).

\part 
Considere um ruído aditivo que trocará o $k$-ésimo bit transmitido, onde $k$ será determinado
por $k=((a+b)\mod 7)$, ou sejam, $z = z_0z_1 \ldots z_k \ldots z_7$, onde todos bits, exceto o bit
$z_k$, são iguais a zero.
(para o exemplo anterior, $k=11 \mod 7 = 4$ e assim o ruído seria $z=0000100$).
Qual será o sinal recebido? Mostre como o decodificador poderá corrigir o erro
calculando a síndrome.

\part
Qual é o tamanho do \emph{codebook} para o código de Hamming (7,4,3)?
Este conjunto é fechado em relação à soma? Mostre um exemplo.

\part 
O que ocorrerá se houverem 2 bits errados? E se houverem 3 bits errados? 

\end{parts}
}

\begin{solution}
\begin{parts}
\part 
  Para o exemplo em que $a=4$ e $b=7$ teríamos 
  $x = 1111111$.

\part 
  $y = x + z = 1111011$

  A síndrome é $s = \mathbf{H} y = (1 0 0)^T$, que corresponde a quinta coluna de $\mathbf{H}$.
  Desta forma determinamos que $z=0000100$, o que está correto. Podemos encontrar $x$ fazendo
  $x = y + z = 1111011 + 0000100 = 1111111$.

\part 
  O codebook possui 16 palavras e é um conjunto fechado em relação à soma.
  
  Exemplo: $w_1 = 0010110$ e $w_2 = 1011010$, $w_1 + w_2 = 1001100$ que satisfaz as equações \ref{eq-mod2-h743a},
  \ref{eq-mod2-h743b} e \ref{eq-mod2-h743c} e portanto é um código de Hamming (7,4,3).

\part 
  Se houverem 2 bits errados o código recebido será não satisfará ao mesmo tempo as equações \ref{eq-mod2-h743a}, 
  \ref{eq-mod2-h743b} e \ref{eq-mod2-h743c}. Será possível determinar a existência de um erro, mas não será possível 
  corrigir corretamente.

  Se houverem 3 bits errados, a palavra recebida será um código de Hamming válido e assim o erro passará despercebido.

\end{parts}
\end{solution}
\end{questions}
