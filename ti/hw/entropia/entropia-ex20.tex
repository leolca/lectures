\subsection{One-time pad (OTP)}

\begin{questions}
\question{
O One-time pad (OTP) é um esquema de criptografia simétrica que utiliza uma chave secreta tão longa quanto a mensagem e é usada apenas uma vez. A segurança do OTP baseia-se na propriedade de que, se a chave é verdadeiramente aleatória, não reutilizada e mantida em segredo, é matematicamente impossível derivar qualquer informação útil sobre a mensagem sem a chave.

Cada bit da mensagem é combinado com o bit correspondente na chave usando a operação XOR (soma módulo 2).
Para descriptografar, a mesma chave é usada novamente, e a operação XOR é aplicada. Como XOR é reversível ($X$ XOR $Y$ XOR $Y$ = $X$), a mensagem original é recuperada.

Seja $X$ a mensagem da fonte e $Y$ a chave OTP, a mensagem cifrada $Z$ é dada por

\begin{equation}
Z = X \oplus Y, 
\end{equation}
ou seja, $Z$ é resultado do XOR entre $X$ e a chave $Y$.

}

\begin{solution}
Um fato fundamental é que o resultado do XOR de uma sequência de bits 
completamente aleatória com uma outra sequência de bits, com qualquer distribuição, 
produz outra sequência de bits completamente aleatória.
Ou seja, se a distribuição de $Y$ é uniforme, $P(Y=1)=P(Y=0)=\frac{1}{2}$,
e $X$ tem uma distribuição qualquer, $P(X=1)=\theta$ e $P(X=0)=1-\theta$,
então, como podemos observar da tabela do XOR (ver abaixo),
\begin{align}
  P(Z=1) &= P(X=1) P(Y=0) + P(X=0) P(Y=1) \\
   & = \theta \frac{1}{2} + (1-\theta) \frac{1}{2} = \frac{1}{2}
\end{align}
e da mesma forma, $P(Z=0) = \frac{1}{2}$, ou seja, independente de 
qual seja a distribuição em $X$, a aleatoriedade máxima em $Y$ garante
que teremos também aleatoriedade máxima em $Z$. 

\begin{center}
\begin{tabular}{lll}
X & Y & Z \\
\hline
0 & 0 & 0 \\
1 & 0 & 1 \\
0 & 1 & 1 \\
1 & 1 & 0  
\end{tabular}
\end{center}

Note ainda que, fixando $Z$, nada podemos saber sobre $X$, uma vez que $Y$ é uniforme.
Para $Z=0$, $X$ pode assumir os valores $0$ ou $1$ com igual probabilidade, já que 
$P(Y=0)=P(Y=1)=\frac{1}{2}$. O mesmo vale para $Z=1$. Desta forma 
\begin{align}
H(X|Z) =& \sum_z p(z) H(X|Z=z) = P(Z=0) H(X|Z=0) + P(Z=1) H(X|Z=1) \\
  =& (P(X=0)P(Y=0)+P(X=1)P(Y=1)) H(X|Z=0) + \\
  &(P(X=1)P(Y=0)+P(X=0)P(Y=1)) H(X|Z=1) \\
  =& \frac{1}{2} H(X|Z=0) + \frac{1}{2} H(X|Z=1) \\
  =& \frac{1}{2} H(X) + \frac{1}{2} H(X) = H(X).
\end{align}
Conhecer assim o valor de $Z$ nada nos diz sobre quem é $X$ (o mesmo vale para $Y$, ou seja, $Z$ não diz nada sobre $Y$).
O OTP permanece seguro quando implementado corretamente com uma chave 
verdadeiramente aleatória e usada apenas uma vez.

\end{solution}

\end{questions}
