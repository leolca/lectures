\subsection{Entropia em um jogo de dados}
% ~/ee/ufsj/2017_02/ti/provas/prova01.tex

\begin{questions}
\question{
Suponha um jogo de dados em que são utilizados dois dados.
Um dado com 6 (seis) lados e outro dado com 4 (quatro) lados.
Seja $X$ e $Y$ as v.a.s associadas ao lançamento de cada um
dos dados, respectivamente, e considere $\mathcal{X}=\{1, \ldots, 6\}$
e $\mathcal{Y}=\{1, \ldots, 4\}$. Sabe-se que os dados utilizados
neste jogo são honestos. Participam do jogo 3 (três) jogadores ($0,1,2$).
Todos jogadores começam com 36 pontos.
Em cada rodada lança-se ambos os dados ao mesmo tempos.
A soma $Z$ dos dados será o quantos pontos um determinado jogador irá perder.
O jogador que perderá ponto é determinado por $W = Z \mod 3$.
Responda/faça o que se pede a seguir.
\begin{parts}
\part Determine $H(Z)$.
\part Determine $H(W)$.
\part Determine $H(Z|W)$.
\part Algum dos jogadores é privilegiado neste jogo? Qual deles? Explique.
\end{parts}
}

\begin{solution}
\begin{parts}
\part 
A v.a. $Z$ possui a seguinte distribuição:
\begin{equation}
Z \sim \left( \frac{1}{24}, \frac{2}{24}, \frac{3}{24}, \frac{4}{24}, \frac{4}{24}, \frac{4}{24}, \frac{3}{24}, \frac{2}{24}, \frac{1}{24} \right) .
\end{equation}

  \begin{equation}
  \begin{matrix}
    2 & 3 & 4 & 5 & 6 & 7 & 8 & 9 & 10 \\
  ( \frac{1}{24} & \frac{2}{24} & \frac{3}{24} & \frac{4}{24} & \frac{4}{24} & \frac{4}{24} & \frac{3}{24} & \frac{2}{24} & \frac{1}{24} )
  \end{matrix}
  \end{equation}
Entropia de $Z$ é dada por
\begin{eqnarray}
H(Z) &=& - \sum_z p(z) \log p(z) \\
        &=& 2 \times \frac{1}{24} \log 24 + 2 \times \frac{1}{12} \log 12 + 2 \times \frac{1}{8} \log 8 + 3 \times \frac{1}{6} \log 6 \\
        &=& \frac{1}{12} (3 + \log 3) + \frac{1}{6} (2 + \log 3) + \frac{1}{4} \times 3 + \frac{1}{2} (1 + \log 3) \\
        &=& \frac{11}{6} + \frac{3}{4} \log 3 .
\end{eqnarray}



\part
\begin{equation}
W \sim \left( \frac{1}{3}, \frac{1}{3}, \frac{1}{3} \right)
\end{equation}

\begin{equation}
H(W) = \log 3 .
\end{equation}


\part 
Iremos utilizar que $H(W|Z) = 0$, pois $W$ é uma função de $Z$.

\begin{eqnarray}
H(Z|W) &=& H(Z,W) - H(W) \\
        &=& H(W|Z) + H(Z) - H(W) \\
        &=& 0 + H(Z) - H(W) \\
        &=& \frac{11}{6} + \frac{3}{4} \log 3 - \log 3 \\
        &=& \frac{11}{6} - \frac{1}{4} \log 3 .
\end{eqnarray}

\part
O jogador associado a $w=2$ é privilegiado pois,
apesar de todos os jogadores possuírem a mesma probabilidade
de perder ponto a cada jogada, o jogador $w=2$ perde em média menos
pontos que os demais. Para verificar, basta analisar $E[Z|W=w]$.

\begin{equation}
E[Z|W=0] = 3 \times \frac{2}{8} + 6 \times \frac{4}{8} + 9 \times \frac{2}{8} = \frac{48}{8} = \frac{24}{4}
\end{equation}

\begin{equation}
E[Z|W=1] = 4 \times \frac{3}{8} + 7 \times \frac{4}{8} + 10 \times \frac{1}{8} = \frac{50}{8} = \frac{25}{4}
\end{equation}

\begin{equation}
E[Z|W=2] = 2 \times \frac{1}{8} + 5 \times \frac{4}{8} + 8 \times \frac{3}{8} = \frac{46}{8} = \frac{23}{4}
\end{equation}


\end{parts}
\end{solution}
\end{questions}
