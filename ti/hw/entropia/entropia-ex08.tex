\subsection{Processamento de Dados}
% ex 21 Cover Thomas

\begin{questions}
\question{
Considere que $X_1 \rightarrow X_2 \rightarrow X_3 \rightarrow \ldots \rightarrow X_n$ seja uma cadeia de Markov nesta ordem,
isto é, 
\begin{equation}
p(x_1, x_2, \ldots, x_n) = p(x_1) p(x_2|x_1) \ldots p(x_n|x_{n-1}).
\end{equation}
Simplifique $I(X_1;X_2;\ldots;X_n)$.
}

\begin{solution}
  Pela regra da cadeia da informação mútua, temos
  \begin{equation}
  I(X_1; X_2; \ldots; X_n) = I(X_1;X_2) + I(X_1;X_3|X_2) + \ldots + I(X_1;X_n|X_2; \ldots, X_{n-2}) .
  \end{equation}
  Pela desigualdade de Markov, o passado e o futuro são independentes, dado o presente. Desta forma,
  todos os termos da equação anterior são nulos, exceto o termo $I(X_1;X_2)$. Teremos assim
  \begin{equation}
  I(X_1; X_2; \ldots; X_n) = I(X_1;X_2) .
  \end{equation}
\end{solution}
\end{questions}
