\subsection{Retirando elementos com e sem reposição}
% ex 14 Cover Thomas

\begin{questions}
\question{
Uma urna contém $r$ bolas vermelhas, $w$ bolas brancas e $b$ bolas pretas.
Qual das opções possui maior entropia? 1) Retirar $k \geq 2$ bolas da urna
com substituição ou 2) sem substituição?

\begin{quote}
Bernoulli para ter sido o primeiro a utilizar o modelo de urnas no estudo das probabilidades.
A inspiração de Bernoulli pode ter sido as loterias, eleições, ou jogos de azar, que envolvem sortear as bolas de um recipiente. Tem sido reconhecido que
nas eleições na época medieval e renascentista de Veneza, incluindo a de Doge\footnote{Doge é a denominação do chefe ou primeiro magistrado eleito, das antigas repúblicas de Gênova e Veneza.}, muitas vezes incluíam a escolha dos eleitores por sorteio, usando bolas de cores diferentes, extraídas de uma urna.
\end{quote}
}

\begin{solution}
  A resposta é simples, pois ao utilizar a estratégia de substituição, teremos que o número de 
  possíveis escolhas em cada etapa será a mesma, ao passo que, ao adotar a estratégia sem substituição,
  a cada iteração teremos o número de possíveis escolhas reduzido e, por conseguinte, teremos menos entropia.

  Para o caso 1) com substituição, teremos $H(X_i \vert X_{i-1}, \ldots, X_1) = H(X_i)$.
  Podemos calcular $H(X_i)$:
  \begin{equation}
  H(X_i) = H \left( \frac{r}{r+w+b}, \frac{w}{r+w+b}, \frac{b}{r+w+b} \right)
  \end{equation}

  Calcular a entropia condicional no caso 2) sem substituição é mais complicado.
  \begin{eqnarray}
  H(X_2 | X_1) &=& \sum_{x} p(x) H(Y|X=x) \nonumber \\
        &=& \frac{r}{r+w+b} H(Y|X=\text{vermelha}) + \frac{w}{r+w+b} H(Y|X=\text{branca}) \nonumber \\
 	&&  + \frac{b}{r+w+b} H(Y|X=\text{preta}) \nonumber \\
        &=& \frac{r}{r+w+b} H \left( \frac{r-1}{r+w+b-1}, \frac{w}{r+w+b-1}, \frac{b}{r+w+b-1} \right) + \nonumber \\
        && \frac{w}{r+w+b} H \left( \frac{r}{r+w+b-1}, \frac{w-1}{r+w+b-1}, \frac{b}{r+w+b-1} \right) + \nonumber \\
        && \frac{b}{r+w+b} H \left( \frac{r}{r+w+b-1}, \frac{w}{r+w+b-1}, \frac{b-1}{r+w+b-1} \right)
  \end{eqnarray}
  Para calcular as entropias condicionais seguintes seria ainda bem mais complicado.


\end{solution}
\end{questions}
