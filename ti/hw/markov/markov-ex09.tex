\begin{questions}
\question{
Foi observada uma língua em que são utilizados três símbolos: A, T, H.
Foram observadas diversas sequências de símbolos produzidas por esta fonte
e constatou-se que, ao longo das sequências, existe uma probabilidade $\frac{1}{2}$ 
de se repetir um dado símbolo. Caso não ocorra a repetição,
haverá igual probabilidade do próximo símbolo ser cada um dos dois remanescentes.
\begin{parts}
\part
Desenhe a cadeia de Markov de primeira ordem que modela este processo estocástico.
\part
Como devemos proceder para determinar a distribuição de estado estacionário?
Qual é a distribuição de estado estacionário?
\part
Qual é a taxa de entropia deste processo?
\end{parts}
}

\begin{solution}
\begin{parts}
\part
\begin{center}
\begin{tikzpicture}[->, >=stealth', auto, semithick, node distance=3cm]
\tikzstyle{every state}=[fill=white,draw=black,thick,text=black,scale=1]
\node[state]    (1)               {$A$};
\node[state]    (2)[right of=1]   {$T$};
\coordinate (Middle) at ($(1)!0.5!(2)$);
\node[state]    (0)[above of=Middle]  {$H$};
\path
(1) edge[bend left,above]  node{$1/4$}  (2)
(1) edge[loop left]        node{$1/2$}  (1)
(1) edge[bend left,left]   node{$1/4$}  (0)
(2) edge[bend left,below]  node{$1/4$}  (1)
(2) edge[loop right]       node{$1/2$}  (2)
(2) edge[bend left,right]  node{$1/4$}  (0)
(0) edge[bend left,left]   node{$1/4$}  (1)
(0) edge[bend left,right]  node{$1/4$}  (2)
(0) edge[loop above]       node{$1/2$}  (0)
;
\end{tikzpicture}
\end{center}


\part
A matriz de transição para a cadeia de Markov $\{Y_n\}$ é
\begin{equation}
\mathbf{P} = 
\begin{blockarray}{cccc}
 & A & T & H \\
\begin{block}{c(ccc)}
 A & 1/2 & 1/4 & 1/4 \\
 T & 1/4 & 1/2 & 1/4 \\
 H & 1/4 & 1/4 & 1/2 \\
\end{block}
\end{blockarray} .
\end{equation}

Para encontrar a distribuição de estado estacionários,
queremos encontrar $\mu^T P = \mu^T$, ou seja, $\mu^T (P - I) = 0$
e iremos denotar $Q = P - I$, assim teremos $\mu^T Q = 0$.
Para incorporar a condução $\sum_i \mu_i = 1$, iremos substituir
a última coluna de $Q$ por uma coluna de uns e substituir o último
elemento do vetor nulo à direita por 1. Teremos assim
\begin{eqnarray}
\mu^T \tilde{Q} &=& \begin{pmatrix} 0 & 0 & 1 \end{pmatrix} \\
\tilde{Q}^T \mu &=& \begin{pmatrix} 0 \\ 0 \\ 1 \end{pmatrix} \\
\begin{pmatrix} 
-1/2 & 1/4 & 1/4 \\
1/4 & -1/2 & 1/4 \\
1 & 1 &  1
\end{pmatrix} 
\begin{pmatrix} \mu_1 \\ \mu_2 \end{pmatrix} 
&=& \begin{pmatrix} 0 \\ 0 \\ 1 \end{pmatrix}
\end{eqnarray}

Devemos então resolver este sistema.
\begin{eqnarray}
\left( \begin{array}{ccc|c}
-1/2 & 1/4  & 1/4 & 0 \\
1/4  & -1/2 & 1/4 & 0 \\
1    &   1  & 1 & 1 \\
\end{array} \right)\\
\left( \begin{array}{ccc|c}
-1/2 & 1/4  & 1/4 & 0 \\
  0  & -3/8 & 3/8 & 0 \\
  0  & 3/2  & 3/2 & 1 \\
\end{array} \right)\\
\left( \begin{array}{ccc|c}
-1/2 & 1/4  & 1/4 & 0 \\
  0  & -3/8 & 3/8 & 0 \\
  0  &   0  & 3   & 1 \\
\end{array} \right)\\
\end{eqnarray}
Temos então que $\mu_3 = \frac{1}{3}$, $\mu_2 = \mu_3$, logo $\mu_2 = \frac{1}{3}$, e
$\frac{1}{2} \mu_1 = \frac{1}{4} \mu_2 + \frac{1}{4} \mu_3$, logo $\mu_1 = \frac{1}{3}$.
Assim, teremos
\begin{equation}
\mu^T = \begin{pmatrix} \frac{1}{3} & \frac{1}{3} & \frac{1}{3} \end{pmatrix}  .
\end{equation}
Pela simetria também podemos concluir que a distribuição de estado estacionário será
$\mu = (1/3 1/3 1/3)$.



Resolvendo com auxílio do Octave, obtemos o mesmo resultados:
\begin{verbatim}
P = [0.5 0.25 0.25; 0.25 0.5 0.25; 0.25 0.25 0.5];
Q = P - eye(3);
Q2=Q; Q2(:,3)=ones(3,1);
[0 0 1]*inv(Q2)
ans =
   0.33333   0.33333   0.33333
\end{verbatim}



\part
A taxa de entropia para um cadeia de Markov de primeira ordem
estacionária será dada da seguinte forma
  \begin{eqnarray}
  H(\mathcal{X}) &=& H'(\mathcal{X}) = \lim_{n \rightarrow \infty} H(X_n \mid X_{n-1}, \ldots, X_1) \\
        &=& \lim_{n \rightarrow \infty} H(X_n \mid X_{n-1}) \\
        && \text{dado que é Markov de 1a ordem} \nonumber \\
        &=& H(X_2 \mid X_1) \quad \text{(estacionário)} \\
        &=& - \sum_{x_2, x_1} p(x_2, x_1) \log p(x_2 \mid x_1) = \sum_i \mu_i \left[ - \sum_j p_{ij} \log p_{ij} \right] \\
        &=& \sum_i \mu_i H( \mathbf{p_i} )
  \end{eqnarray}
onde $\mu$ é a distribuição estacionária,
$p_{ij}$ a probabilidade de transição de $i$ para $j$ (elementos da matriz $P$)
e $\mathbf{p_i}$ a $i$-ésima linha da matriz $P$ (as probabilidades de transição à partir
do estado $i$).

Para o nosso exemplo, teremos
\begin{eqnarray}
H(\mathcal{X}) &=& \mu_1 H(\mathbf{p_1}) + \mu_2 H(\mathbf{p_2}) + \mu_3 H(\mathbf{p_3}) = 3 \times \frac{1}{3} H(\mathbf{p_1}) \\
        &=& H\left( \frac{1}{2}, \frac{1}{4}, \frac{1}{4} \right) \\
        &=& \frac{1}{2} + 2 \times \frac{1}{4} \times 2 = \frac{3}{2}  \textmd{\ bits} .
\end{eqnarray}


\end{parts}
\end{solution}
\end{questions}
