\subsection{Modelo Genético}
% https://www.dartmouth.edu/~chance/teaching_aids/books_articles/probability_book/Chapter11.pdf

\begin{questions}
\question{
O modelo genético simples de herança de traços (ou fenótipo) 
em animais consiste em considerar que estes traços são determinados por pares
de genes. Neste modelo, cada um dos genes de um par pode ser de dois tipos G ou g. 
Um indivíduo pode ter uma das seguintes combinações: GG, Gg (que é geneticamente
equivalente a gG) ou gg. Usualmente os tipos GG e Gg são indistinguíveis na aparência.
Desta forma, dizemos que o gene G domina o gene g. Um indivíduo é chamado \textit{dominante} 
se possuir os genes GG, \textit{híbrido} se possuir os genes Gg e \textit{recessivo} se
possuir os genes gg.

No acasalamento de dois animais, a prole herda um gene de cada par de cada um dos seus progenitores.
A suposição básica em genética é que estes genes são escolhidos aleatoriamente e de forma 
independente um do outro. Esta suposição determina a probabilidade de ocorrência de cada 
um dos tipos de prole. A prole de dois pais puramente dominantes deverá ser dominante,
a prole de dois pais recessivos deverá ser recessiva, e a prole de um dominante e um recessivo
deverá ser híbrida. 

Vamos considerar agora um processo de reprodução continuada, onde a cada etapa há o acasalamento
de dois indivíduos produzindo uma nova prole. Para simplificar, vamos inicialmente
considerar que um indivíduo qualquer se acasalará com um indivíduo híbrido.
No processo de acasalamento de um híbrido com um dominante, a prole necessariamente terá um
gene G proveniente do dominante e poderá receber um gene G ou um gene g do indivíduo híbrido,
com igual probabilidade. Ou seja, a prole será do tipo GG com probabilidade 0.5, Gg com probabilidade 0.5
e gg com probabilidade 0.
Considerando agora o acasalamento entre um híbrido e um recessivo, a prole poderá ser
Gg com probabilidade 0.5 e gg com probabilidade 0.5.
Por fim, no acasalamento de dois híbridos, a prole poderá ser GG com probabilidade 0.25,
Gg com probabilidade 0.5 e gg com probabilidade 0.25.

Este processo de acasalamento com um híbrido pode ser descrito através de uma cadeia de Markov,
onde os estados são o gene da prole, podendo ser GG, Gg e gg. Este processo estocástico é homogêneo,
então a cadeia de Markov poderá ser descrita pela seguinte matriz
\begin{equation}
\mathbf{P} = 
\begin{blockarray}{cccc}
 & GG & Gg & gg \\
\begin{block}{c(ccc)}
 GG & 0.5  & 0.5 & 0    \\
 Gg & 0.25 & 0.5 & 0.25 \\
 gg & 0    & 0.5 & 0.5  \\
\end{block}
\end{blockarray} .
\end{equation}

Suponha que agora o processo continuado de acasalamento seja feito sempre com um
indivíduo dominante. Neste caso, a matriz de transição será
\begin{equation}
\mathbf{P} = 
\begin{blockarray}{cccc}
 & GG & Gg & gg \\
\begin{block}{c(ccc)}
 GG & 1   & 0   & 0  \\
 Gg & 0.5 & 0.5 & 0  \\
 gg & 0   & 1   & 0  \\
\end{block}
\end{blockarray} .
\end{equation}

Vamos supor agora um modelo mais realista. Suponha que temos dois animais, de sexo
oposto, e suponha que os traço seja independente do sexo. Vamos iniciar o processo
com dois indivíduos de sexo oposto, este casal irá reproduzir. Vamos selecionar
dois descendentes e acasalar estes dois, continuando indefinidamente o processo.

Agora, cada estado será representado por um par de indivíduos. Os estados do processo
serão: $s_1 = (GG,GG)$, $s_2 = (GG,Gg)$, $s_3 = (GG,gg)$, $s_4 = (Gg,Gg)$, $s_5 = (Gg,gg)$ e
$s_6 = (gg,gg)$.

\begin{parts}
\part Calcule as probabilidades de transição e encontre a matriz de transição $P$.
\part Represente a cadeia de Markov através de um grafo.
\part Determine a distribuição de estado estacionário.
\part Calcule a taxa de entropia do processo.
\end{parts}

}

\begin{solution}
Vamos ilustrar o cálculo das probabilidades de transição a partir do estado $s_2$.
Quando o processo se encontra neste estado, um dos progenitores é do tipo GG e o
outro é do tipo Gg. A probabilidade de gerar uma prole dominante é de 1/2.
Desta forma, a probabilidade de transição para $s_1$ será de 1/4, a probabilidade
de transição para $s_2$ será de 1/2 e a probabilidade de transição para $s_4$
será de 1/4.


A matriz de transição será da seguinte forma:

\begin{equation}
\mathbf{P} = 
\begin{blockarray}{ccccccc}
 & GG,GG & GG,Gg & GG,gg & Gg,Gg & Gg,gg & gg,gg \\
\begin{block}{l(cccccc)}
 GG,GG & 1      & 0    & 0     & 0    & 0    & 0      \\
 GG,Gg & 0.25   & 0.5  & 0     & 0.25 & 0    & 0      \\
 GG,gg & 0      & 0    & 0     & 1    & 0    & 0      \\
 Gg,Gg & 0.0625 & 0.25 & 0.125 & 0.25 & 0.25 & 0.0625 \\ 
 Gg,gg & 0      & 0    & 0     & 0.25 & 0.5  & 0.25   \\
 gg,gg & 0      & 0    & 0     & 0    & 0    & 1      \\
\end{block}
\end{blockarray} .
\end{equation}



\begin{center}
\begin{tikzpicture}[->, >=stealth', auto, semithick, node distance=3cm]
\tikzstyle{every state}=[fill=white,draw=black,thick,text=black,scale=1]
\node[state]    (s4)                    {$s_4$};
\node[state]    (s1)[above left of=s4]  {$s_1$};
\node[state]    (s2)[left of=s4]        {$s_2$};
\node[state]    (s3)[below of=s4]       {$s_3$};
\node[state]    (s5)[right of=s4]       {$s_5$};
\node[state]    (s6)[above right of=s4] {$s_6$};
\path
(s2) edge[loop left]         node{$\frac{1}{2}$}    (s2)
(s5) edge[loop right]        node{$\frac{1}{2}$}    (s5)
(s4) edge[loop above]        node{$\frac{1}{4}$}    (s4)
(s1) edge[loop left]         node{$1$}      (s1)
(s6) edge[loop right]        node{$1$}      (s6)
(s2) edge[bend left,above]   node{$\frac{1}{4}$}    (s4)
(s4) edge[bend left,below]   node{$\frac{1}{4}$}    (s2)
(s2) edge[bend left,left]    node{$\frac{1}{4}$}    (s1)
(s4) edge[bend right,above right]   node{$\frac{1}{16}$}    (s1)
(s4) edge[bend left,above left]   node{$\frac{1}{16}$}    (s6)
(s4) edge[bend right,left]   node{$\frac{1}{8}$}    (s3)
(s3) edge[bend right,right]  node{$1$}      (s4)    
(s4) edge[bend left,above]   node{$\frac{1}{4}$}    (s5)
(s5) edge[bend left,below]   node{$\frac{1}{4}$}    (s4)
(s5) edge[bend right,right]  node{$\frac{1}{4}$}    (s6)
;  
\end{tikzpicture}
\end{center}


\begin{lstlisting}[language=Octave]
P = [   1 0 0 0 0 0; ...
        0.25 0.5 0 0.25 0 0; 
        0 0 0 1 0 0; ...
        0.0625 0.25 0.125 0.25 0.25 0.0625; ...
        0 0 0 0.25 0.5 0.25; ...
        0 0 0 0 0 1];
Q = P - eye(size(P));
Q1 = Q;
Q1(:,end)=ones(size(P,1),1);
det(Q1)
ans = 0
% faríamos como anteriormente, 
% [zeros(1,size(P,2)-1), 1] * inv(Q1)
% mas a matriz Q eh singular
% vamos utilizar então a pseudo-inversa
[zeros(1,size(P,2)-1), 1] * pinv(Q1)
ans =

   0.50000   0.00000  -0.00000  -0.00000   0.00000   0.50000
% simulacao
mu=rand(1,6); mu=mu/sum(mu); for i=1:1E6, mu = mu*P; end; disp(mu)
   0.47487   0.00000   0.00000   0.00000   0.00000   0.52513

% utilizando a função criada no exercício anterior
H = markovEntropyRate(P);
% H =   -2.2898e-16
\end{lstlisting}

Podemos verificar que não existe solução única para este problema,
o que fica explicito quando verificamos que a matriz $P-I$
é singular, não possuindo assim inversa.
Qualquer solução da forma $\mu^T = (\mu_1, 0, 0, 0, 0, \mu_6)$
com $\mu_1 + \mu_6 = 1$ é uma solução válida, o que pode ser
facilmente verificado realizando a multiplicação $\mu^T P$.


Note que, resolver $\mu^T P = \mu^T$ é equivalente a resolver $P^T \mu = \mu$,
ou seja, buscamos o autovetor de $P^T$ associado ao autovalor $1$.
Para este caso, o autovalor $1$ possui multiplicidade $2$, e assim
observamos 2 autovetores associados ao autovalor 1.
\begin{lstlisting}[language=Octave]
[V, lambda] = eig(P')
V =

   1.00000   0.00000   0.00968   0.55780  -0.13608  -0.31623
   0.00000   0.00000  -0.26548  -0.32553   0.54433   0.63246
   0.00000   0.00000  -0.34752  -0.06217  -0.27217  -0.00000
   0.00000   0.00000   0.85912  -0.40238  -0.54433  -0.00000
   0.00000   0.00000  -0.26548  -0.32553   0.54433  -0.63246
   0.00000   1.00000   0.00968   0.55780  -0.13608   0.31623
lambda =

Diagonal Matrix

   1.00000         0         0         0         0         0
         0   1.00000         0         0         0         0
         0         0  -0.30902         0         0         0
         0         0         0   0.80902         0         0
         0         0         0         0   0.25000         0
         0         0         0         0         0   0.50000
\end{lstlisting}
Se denotarmos por $v_1 = (1,0,0,0,0,0)^T$ e $v_2 = (0,0,0,0,0,1)^T$,
teremos que qualquer combinação da forma $\mu^T = \alpha v_1 + (1-\alpha)v_2$,
para $0 \leq \alpha \leq 1$, será uma solução para o problema.


Conforme verificamos acima, na solução computacional, uma possível
distribuição de estado estacionário é
\begin{equation}
\mu^T = \left( \frac{1}{2}, 0, 0, 0, 0, \frac{1}{2} \right) .
\end{equation}

Podemos calcular facilmente a taxa de entropia.
\begin{eqnarray}
  H(\mathcal{X}) &=& \sum_i \mu_i H( \mathbf{p_i} ) \nonumber \\
        &=& \mu_1 H(1,0,0,0,0,0) + 0 + 0 + 0 + 0 + \mu_6 H(0,0,0,0,0,1) = 0 .
\end{eqnarray} 

\end{solution}
\end{questions}
