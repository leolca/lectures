\subsection{Cadeia de Markov}
% harvard / HW2_ES250_sol_a

\begin{questions}
\question{
Seja $X_1$ uniformemente distribuído sobre os estados $\{0,1,2\}$. Seja $\{X_i\}_{1}^{\infty}$ 
uma cadeia de Markov com matriz de transição $P$, ou seja, $P(X_{n+1}=j \vert X_n = i) = P_{ij}$, $i,j \in \{0,1,2\}$.
\begin{equation}
        P = [P_{ij}] = 
        \begin{bmatrix} 
        \frac{1}{2} & \frac{1}{4} & \frac{1}{4} \\
        \frac{1}{4} & \frac{1}{2} & \frac{1}{4} \\
        \frac{1}{4} & \frac{1}{4} & \frac{1}{2} 
        \end{bmatrix}
\end{equation}

\begin{parts}
\part $\{X_n\}$ é estacionária?

\part Calcule $\lim_{n \rightarrow \infty}\frac{1}{n}H(X_1, \ldots, X_n)$. 

\part
Considere agora um processo derivado $Z_1, Z_2, \ldots, Z_n$ onde
\begin{eqnarray}
        Z_1 = X_1 \nonumber & \\
        Z_i = (X_i - X_{i-1})\mod{3} , & \quad  i=2, \ldots, n
\end{eqnarray}
Desta forma $Z^n$ codifica as transições ao invés dos estados.

Encontre $H(Z_1,Z_2,\ldots,Z_n)$.

\part 
Encontre $H(Z_n)$ e $H(X_n)$, para $n \geq 2$.

\part
Encontre $H(Z_n \vert Z_{n-1})$, para $n \geq 2$.

\part
$Z_{n-1}$ e $Z_{n}$ são independentes para $n \geq 2$?

\end{parts}
}

\begin{solution}
\begin{parts}
\part 
Seja $\mu_n$ a função massa de probabilidade no instante $n$. No instante $n=1$
temos $\mu_1 = (1/3, 1/3, 1/3)$. No instante seguinte, teremos

\begin{eqnarray}
\mu_2 &=& \mu_1 P \\
        &=& \left( \frac{1}{3} \times \frac{1}{2} + \frac{1}{3} \times \frac{1}{4} + \frac{1}{3} \times \frac{1}{4} , \frac{1}{3} \times \frac{1}{4} + \frac{1}{3} \times \frac{1}{2} + \frac{1}{3} \times \frac{1}{4}, \frac{1}{3} \times \frac{1}{4} + \frac{1}{3} \times \frac{1}{4} + \frac{1}{3} \times \frac{1}{2}  \right) \nonumber \\
        &=& \left( \frac{1}{3}, \frac{1}{3}, \frac{1}{3} \right) = \mu_1 .
\end{eqnarray}

Da mesma forma podemos constatar que $\mu_n = (1/3, 1/3, 1/3)$ para todo $n$ e assim
$\{X_n\}$ é estacionário.


\part
Como $\{X_n\}$ é uma cadeia de Markov estacionária, teremos
\begin{eqnarray} 
\lim_{n \rightarrow \infty}\frac{1}{n}H(X_1, \ldots, X_n) &=& H(X_2 | X_1) \nonumber \\
        &=& \sum_{k=0}^{2} \Pr(X_1 = k) H(X_2 | X_1 = k) \nonumber \\
        &=& 3 \times \frac{1}{3} \times H\left( \frac{1}{2}, \frac{1}{4}, \frac{1}{4} \right) \nonumber \\
        &=& \frac{3}{2} .
\end{eqnarray}


\part
Como $X \in \{0,1,2\}$, teremos que existe uma função injetora de $(X_1,X_2,\ldots,X_n)$ em
$(Z_1,Z_2,\ldots,Z_n)$ (eles são \textit{one-to-one}). Desta forma, fazendo uso da
regra da cadeia da entropia, podemos escrever
\begin{eqnarray} 
H(Z_1,Z_2,\ldots,Z_n) &=& H(X_1,X_2,\ldots,X_n) \nonumber \\
        &=& \sum_{k=1}^{n} H(X_k | X_1, \ldots X_{k-1}) \nonumber \\
        &=& H(X_1) + \sum_{k=2}^{n} H(X_k | X_{k-1}) \nonumber \\
        &=& H(X_1) + (n-1) H(X_2 | X_1) \nonumber \\
        &=& \log 3 + (n-1) \times \frac{3}{2}
\end{eqnarray}

\part 
Como $\{X_n\}$ é estacionário com $\mu_n = (1/3, 1/3, 1/3)$,
\begin{equation}
H(X_n) = H(X_1) = H\left( \frac{1}{3}, \frac{1}{3}, \frac{1}{3} \right) = \log 3 .
\end{equation}

Teremos $Z_n = 0$ quando $(X_n = 0, X_{n-1} = 0)$, $(X_n = 1, X_{n-1} = 1)$,
$(X_n = 2, X_{n-1} = 2)$, assim
$\Pr(Z_n = 0) = \frac{1}{3} \times \frac{1}{2} + \frac{1}{3} \times \frac{1}{2} + \frac{1}{3} \times \frac{1}{2} = \frac{1}{2}$.

Teremos $Z_n = 1$ quando $(X_n = 1, X_{n-1} = 0)$, $(X_n = 2, X_{n-1} = 1)$,
$(X_n = 0, X_{n-1} = 2)$, assim
$\Pr(Z_n = 1) = \frac{1}{3} \times \frac{1}{4} + \frac{1}{3} \times \frac{1}{4} + \frac{1}{3} \times \frac{1}{4} = \frac{1}{4}$.

Teremos $Z_n = 2$ quando $(X_n = 2, X_{n-1} = 0)$, $(X_n = 0, X_{n-1} = 1)$,
$(X_n = 1, X_{n-1} = 2)$, assim
$\Pr(Z_n = 2) = \frac{1}{3} \times \frac{1}{4} + \frac{1}{3} \times \frac{1}{4} + \frac{1}{3} \times \frac{1}{4} = \frac{1}{4}$.

Para $n \geq 2$ temos
\begin{equation}
Z_n = \begin{cases} 
0, \quad \frac{1}{2}; \\
1, \quad \frac{1}{4}; \\
2, \quad \frac{1}{4}.
\end{cases}
\end{equation}
Desta forma, $H(Z_n) = H\left( \frac{1}{2}, \frac{1}{4}, \frac{1}{4} \right) = \frac{3}{2}$.



\part 
Devido à simetria de $P$, $\Pr(Z_n | Z_{n-1}) = \Pr(Z_n)$ para $n \geq 2$.
Pois, podemos verificar que $Z_n$ depende apenas de $X_n$ e $X_{n-1}$, sendo
independente de $X_{n-2}$.

\begin{center}
\begin{tikzpicture}[->, >=stealth', auto, semithick, node distance=3cm]
\tikzstyle{every state}=[fill=white,draw=black,thick,text=black,scale=1]
\node[state]    (1)               {$1$};
\node[state]    (2)[right of=1]   {$2$};
\coordinate (Middle) at ($(1)!0.5!(2)$);
\node[state]    (0)[above of=Middle]  {$0$};
\path
(1) edge[bend left,above]  node{$1/4$}  (2)
(1) edge[loop left]        node{$1/2$}  (1)
(1) edge[bend left,left]   node{$1/4$}  (0)
(2) edge[bend left,below]  node{$1/4$}  (1)
(2) edge[loop right]       node{$1/2$}  (2)
(2) edge[bend left,right]  node{$1/4$}  (0)
(0) edge[bend left,left]   node{$1/4$}  (1)
(0) edge[bend left,right]  node{$1/4$}  (2)
(0) edge[loop above]       node{$1/2$}  (0)
;
\end{tikzpicture}
\end{center}

Desta forma, $H(Z_n | Z_{n-1}) = H(Z_n) = \frac{3}{2}$.



\part 
Sim, pois $H(Z_n | Z_{n-1}) = H(Z_n)$.

\end{parts}
\end{solution}
\end{questions}

