\subsection{Conjunto típico}

\begin{questions}
\question{
Considere $M$ a soma dos algarismos do seu número de matrícula
e $m = 4 + M \mod 3$. Por exemplo, se o número de matrícula é 0123456789, teremos
$M = 0 + 1 + 2 + 3 + 4 + 5 + 6 + 7 + 8 + 9 = 45$ e $m = 4 + 45 \mod 3 = 4 + 0 = 4$. 

Suponha $X_{1:n}$ uma sequência de comprimento $n$, de v.a. i.i.d., em um alfabeto
$\mathcal{X} = \{1,2,\ldots,m\}$, com distribuição $p$ dada 
\begin{equation}
p(X = x) = \begin{cases}
           \frac{1}{2}   & x = 1, \\
           \frac{1}{2^2} & x = 2, \\
           \vdots        &      \\
           \frac{1}{2^{m-2}} & x = m-2 \\
           \frac{1}{2^{m-1}} & x = m-1 \textmd{ e } x=m.
           \end{cases}
\end{equation}
Seja $A_{\epsilon}^{(n)}$ o conjunto típico e considere $\epsilon=0.1$ e $n=16$.

\begin{parts}
\part 
Determine o limite superior e inferior para o tamanho do conjunto típico.

\part 
Qual é a sequência mais provável e qual é a menos provável?

\part
Quantos tipos existem?

\part 
Qual é o tipo mais provável? Dê um exemplo de sequência deste tipo.

\part 
Qual é o limite superior e inferior para o tamanho da classe de tipo mais provável?

\part
Qual é o tamanho da classe de tipo mais provável? Este valor está dentro dos limites encontrados no item anterior?
\end{parts}
}

\begin{solution}

\begin{parts}
\part 

\begin{lstlisting}[language=Octave]
 m = 4; n = 16; eps = 0.1;
 p = 2.^(-[1:m]);
 p(m)=p(m-1);
\end{lstlisting}


\begin{lstlisting}[language=Octave]
 function H = entropy(p), id=find(p==0); p(id)=[];  H=-sum(p.*log2(p)); endfunction
 H = entropy(p)
 L = ceil(2^(n*(H+eps)))
 l = floor((1-eps)*2^(n*(H-eps)))
 2^(n*H)
\end{lstlisting}

\part 
Mais provável: sequência com apenas 1.\\
Menos provável: sequência em que $x$ seja apenas $m$ ou $m-1$.

\part 
\begin{lstlisting}[language=Octave]
nchoosek (n + m - 1, m - 1)
\end{lstlisting}


\part
Aquele com a distribuição mais próximo da distribuição subjacente.

\begin{verbatim}
m=4 (1 (8), 2 (4), 3 (2), 4 (2)) 
     ex: [1,2,1,4,1,2,1,1,2,2,4,3,1,1,1,3]
[8, 4, 2, 2]
demais: (1 (8), 2 (4), 3 (2), 4 (1), 5 (1)) 
     ex: [1,2,1,4,1,2,1,1,2,2,5,3,1,1,1,3]
     [8, 4, 2, 1, 1], [8, 4, 2, 1, 1, 0], [8, 4, 2, 1, 1, 0, 0]
P = [8, 4, 2, 2]/n
P = [8, 4, 2, 1, 1]/n
\end{verbatim}

\part
\begin{lstlisting}[language=Octave]
HP = entropy(P)
LT = 2^(n*HP)
lT = 2^(n*HP)/((n+1)^m)
\end{lstlisting}


\part
\begin{lstlisting}[language=Octave]
function c = multicoeff (n,m), 
%  Compute the multinomial coefficient using product of binomials
%
%       /           \ 
%       |     n     | 
%  c =  | k1 ... km |
%       \           /
%
%  c = multicoeff (k)
%  where k is an array k = [k1 k2 ... km] and 
%  n is simplicity given by the sum: n = sum(k)
% 
%  c = multicoeff (n,m)
%  computes all multinomials at level n
%  (m=2 for binomials, m=3 for trinomials, etc)

  if nargin < 2,
    k = n;
    c=1; 
    for i=1:length(k), 
      c*=bincoeff(sum(k(1:i)),k(i)); 
    endfor;
  else
    r = npermutek([0:n],m);
    id = find ( sum(r,2) == n);
    r = r(id,:);
    c = [];
    for i = 1:size(r,1),
      c(i) = multicoeff (r(i,:));
    endfor
    c = c';
  endif 
endfunction


multicoeff (P*n)
\end{lstlisting}

\end{parts}
\end{solution}
\end{questions}
