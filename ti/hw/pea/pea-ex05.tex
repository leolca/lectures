\subsection{Código genético do DNA}

\begin{questions}
\question{
        O código genético do DNA pode ser representado por uma 
        sequência constituídas por letras de um alfabeto
        $\mathcal{X} = \{A,T,C,G\}$, que representam os 4 tipos
        de bases nitrogenadas (Adenina, Timina, Citosina e Guanina).
        Verificou-se que as sequências de comprimento 4 são relevantes
        nas funções regulatórias dos genes.

        Analisando estas sequências de comprimento 4 sob a ótica do método de tipos,
        considerando que as probabilidades das bases hidrogenadas são dadas por
        $q = (\nicefrac{3}{8}, \nicefrac{3}{8}, \nicefrac{1}{8}, \nicefrac{1}{8})$, respectivamente,
        e assumindo a independência na ocorrência dessas bases hidrogenadas,
        responda às questões abaixo.
\begin{parts}
\part
Quantos tipos existem?

\part
    Quantas sequências do tipo $P$ abaixo existem?
    \begin{equation}
     P_{x_{1:4}} = \left( \frac{3}{4}, \frac{1}{4}, 0, 0 \right)
    \end{equation}

\part
Qual é a probabilidade da classe de tipo para o tipo
$P_{x_{1:4}} = \left( \nicefrac{3}{4}, \nicefrac{1}{4}, 0, 0 \right)$?

\end{parts}
}

\begin{solution}
\begin{parts}
\part
    \begin{equation}
    \vert \mathcal{P}_n \vert = {n + \vert \mathcal{X} \vert - 1 \choose \vert \mathcal{X} \vert - 1} = {7 \choose 3} = \frac{7!}{3! 4!} = 35
    \end{equation}

\part
    As sequências onde há 3 ocorrências de A e 1 ocorrência de T são
    apenas 4, a saber: TAAA, ATAA, AATA, AAAT.

    Podemos também utilizar o coeficiente multinomial
    \begin{equation}
    {4 \choose {3, 1, 0, 0}} = \frac{4!}{3! 1! 0! 0!} = 4 .
    \end{equation}

\part
    \begin{eqnarray}
    Q^n(T(P)) &=& |T(P)| q_A^3 q_T = 4 (\nicefrac{3}{8})^3 (\nicefrac{3}{8}) \\
                &=& 4 (\nicefrac{3}{8})^4
    \end{eqnarray}

 
\end{parts}
\end{solution}
\end{questions}
