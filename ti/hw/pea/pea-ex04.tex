\subsection{Jogo de moeda}

\begin{questions}
\question{
Em um determinado jogo de moeda, uma moeda honesta é lançada 100 vezes.

\begin{parts}
\part
Estime o limite superior da probabilidade de observarmos uma sequências com
   \begin{inlineenumroman}
   \item 60 caras,
   \item 80 caras.
   \end{inlineenumroman}

\part 
Mostre como podemos calcular explicitamente a probabilidade de obtermos uma sequência com
   \begin{inlineenumroman}
   \item 60 caras,
   \item 80 caras
   \end{inlineenumroman}
(não é necessário realizar os cálculos).

\part 
Mostre também como podemos calcular a probabilidade de obtermos uma sequência
em que o número de caras esteja entre 40 e 60 (não é necessário realizar os cálculos).

\part
Escreva um algoritmo (Octave/Matlab ou pseudo-código) para calcular esta probabilidade.

\part
Qual é o problema em se realizar o cálculo explícito (neste caso, justificando a
utilização do limite)?
\end{parts}
}

\begin{solution}
\begin{parts}
\part
A probabilidade de uma sequência depende apenas do tipo.
Um limite superior para a probabilidade da classe de tipo para um
determinado tipo $P$, considerando uma distribuição subjacente $Q$,
é dada por
\begin{equation}
 Q^n (T(P)) \leq 2^{-n D(P||Q)} .
 \end{equation}
 Considerando a moeda honesta, temos que $Q = (\frac{1}{2}, \frac{1}{2})$.
 Para o item a) temos o tipo $P_a = (0.6, 0.4)$ e para o item b) temos
 o tipo $P_b = (0.8, 0.2)$. Devemos então calcular as divergências
 \begin{eqnarray}
 D(P_a||Q) &=& \frac{3}{5} \log \left( \frac{3}{5} \times 2 \right) + \frac{2}{5} \log \left( \frac{2}{5} \times 2 \right) \nonumber \\
        &=& \frac{3}{5} \left( 1 + \log 3 - \log 5 \right) + \frac{2}{5} \left( 2 - \log 5 \right) \nonumber \\
        &=& \frac{7}{5} + \frac{3}{5} \log 3 - \log 5  \approx 0.029049
\end{eqnarray}
\begin{equation}
 Q^n (T(P_a)) \leq 2^{-n D(P_a||Q)} \approx 2^{-2.9049} \approx 0.13352 .
\end{equation}

\begin{eqnarray}
D(P_b||Q) &=& \frac{4}{5} \log \left( \frac{4}{5} \times 2 \right) + \frac{1}{5} \log \left( \frac{1}{5} \times 2 \right) \nonumber \\
        &=& \frac{4}{5} \left( 3 - \log 5 \right) + \frac{1}{5} \left( 1 - \log 5 \right) \nonumber \\
        &=& \frac{13}{5} - \log 5  \approx 0.27807
\end{eqnarray}
\begin{equation}
 Q^n (T(P_b)) \leq 2^{-n D(P_b||Q)} \approx 2^{-27.807} \approx 4.2585 \times 10^{-9} .
\end{equation}

\begin{lstlisting}[language=Octave]
function D = kldivergence(p,q)
D = sum(p.*log2(p./q));
endfunction

D = kldivergence([0.6 0.4],[0.5 0.5]);
2^(-100*D)
ans =  0.13351

D = kldivergence([0.8 0.2],[0.5 0.5]);
2^(-100*D)
ans =    4.2580e-09
\end{lstlisting}



\part
Explicitamente, a probabilidade da classe de tipo é igual à probabilidade de uma sequência deste dado
tipo vezes o número de sequências deste dado tipo,
\begin{equation}
Q^n (T(P_a)) = {100 \choose 60} p^{60} (1-p)^{40} = \frac{100!}{60! 40!} \left( \frac{1}{2} \right)^{100} .
\end{equation}
De forma semelhante,
\begin{equation}
Q^n (T(P_a)) = {100 \choose 80} p^{80} (1-p)^{20} = \frac{100!}{80! 20!} \left( \frac{1}{2} \right)^{100} .
\end{equation}

\part 
A probabilidade de obtermos uma sequência em que o número de caras esteja entre 40 e 60
é dada por
\begin{equation}
P_{40 \leq k \leq 60} = \sum_{k=40}^{60} {100 \choose k} p^{k} (1-p)^{100-k} = \sum_{k=40}^{60} {100 \choose k} \left( \frac{1}{2} \right)^{100}
\end{equation}

\begin{lstlisting}[language=Octave]
n=100; p=0; for k = 40:60,  p+=nchoosek(n,k)*(0.5)^n; end;
p =  0.96480
\end{lstlisting}



\part
Para realizar o cálculo explícito das probabilidades haverá problema de precisão numérica
quando o valor de $n$ for grande, além do alto custo computacional,
não sendo possível realizar de forma prática os
cálculos, justificando assim a utilização do limite para estes casos.

\begin{lstlisting}[language=Octave]
tic; n=10E4; p=0; 
for k = 0.4*n:0.6*n,  p+=nchoosek(n,k)*(0.5)^n; end; 
p, toc
p = NaN
Elapsed time is 14.2635 seconds.
\end{lstlisting}

\end{parts}
\end{solution}
\end{questions}
