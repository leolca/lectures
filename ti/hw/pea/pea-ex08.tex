\subsection{Conjunto típico}

\begin{questions}
\question{
Considere uma sequência i.i.d. de variáveis aleatórias $X_1, \ldots, X_n$ em um alfabeto ternário
$\mathcal{X} = \{0,1,2\}$ com distribuição $p = (0.6, 0.3, 0.1)$.
\begin{parts}
\part 
Calcule $H(X)$.

\part
  Considere $n=10$, $n=50$, $n=100$, $n=200$, $n=500$ e $n=1000$ e $\epsilon = 0.1$. 
  Quais sequências pertencem ao conjunto típico $A^{(n)}_{\epsilon}$? (Ou seja, quais tipos)?
  Quantos tipos existem no conjunto típico?
  Qual é a probabilidade do conjunto típico? Qual percentual do conjunto de todas as sequências está dentro do conjunto típico?
  Calcule os limites superiores e inferiores para: i) o tamanho do conjunto típico; e ii) a probabilidade do conjunto típico.
  Faça uma tabela para comparar os resultados para cada um dos valores de $n$.
  Verifique se o tamanho do conjunto típico está dentro dos limiares dados na teoria.
  (dica veja em \url{https://en.wikipedia.org/wiki/Multinomial_theorem} uma forma computacionalmente eficiente de calcular o coeficiente multinomial).

\part
  Para $n=10$ e $n=20$, faça um gráfico indicando o percentual das sequências do conjunto de todas as sequências que estão em cada uma das
  classes de tipo contidas no conjunto típico.


\end{parts}
}


\begin{solution}
\begin{parts}
\part
  \begin{lstlisting}[language=Octave]
  p = [0.6 0.3 0.1];
  H = -sum(p.*log2(p));
  H = 1.2955 bits
  \end{lstlisting}

\part
  \begin{lstlisting}[language=Octave]
  pxn = p1^k1 * p2^k2 * p3^k3  % onde p1, p2 e p3 são as probabilidades dos símbolos e k1, k2 e k3 o número de ocorrência
  \end{lstlisting}

  \begin{lstlisting}[language=Octave]
  eps=0.1; n=10;
  count=0; 
  for k1=0:n, for k2=0:n-k1, 
     pn=p(1)^k1*p(2)^k2*p(3)^(n-k1-k2); 
     lpn=-(1/n)*log2(pn); 
     if abs(lpn-H)<eps, 
        count++; 
        printf('%d %d %d\n',k1,k2,n-k1-k2); 
     endif; 
  endfor; endfor;
  \end{lstlisting}

  \begin{lstlisting}[language=Octave]
  count=0; 
  for k1=0:n, for k2=0:n-k1, 
      k=[k1, k2, n-k1-k2]; 
      lpn=-(1/n)*(sum(k.*log2(p))); 
      if abs(lpn-H)<eps, 
         count++; 
         printf('%d %d %d\n',k); 
      endif; 
  endfor; endfor; 
  \end{lstlisting}

  \begin{lstlisting}[language=Octave]
  % https://en.wikipedia.org/wiki/Multinomial_theorem#Multinomial_coefficients
  function c=multicoeff (k), 
      c=1; 
      for i=1:length(k), 
          c*=bincoeff(sum(k(1:i)),k(i)); 
      endfor; 
  endfunction
  \end{lstlisting}

  \begin{lstlisting}[language=Octave]
  n=10; PA=0; count=0; 
  for k1=0:n, for k2=0:n-k1, 
      k=[k1, k2, n-k1-k2]; 
      lpn=-(1/n)*(sum(k.*log2(p))); 
      pn=p(1)^k1*p(2)^k2*p(3)^(n-k1-k2); 
      if abs(lpn-H)<eps, 
         count++; 
         printf('%d %d %d\n',k); 
         PA+=pn*multicoeff(k); 
      endif; 
  endfor; endfor;
  \end{lstlisting}

\begin{lstlisting}[language=Octave]
octave:81> H
H =  1.2955
octave:82> eps=0.1;
octave:83> n=500; PA=0; count=0; 
for k1=0:n, for k2=0:n-k1, k=[k1, k2, n-k1-k2]; 
lpn=-(1/n)*(sum(k.*log2(p))); pn=p(1)^k1*p(2)^k2*p(3)^(n-k1-k2); 
if abs(lpn-H)<eps, count++; PA+=pn*multicoeff(k); endif; endfor; endfor; 
PA
PA =  0.99420
octave:84> n=200; PA=0; count=0; for k1=0:n, for k2=0:n-k1, 
k=[k1, k2, n-k1-k2]; lpn=-(1/n)*(sum(k.*log2(p))); pn=p(1)^k1*p(2)^k2*p(3)^(n-k1-k2); 
if abs(lpn-H)<eps, count++; PA+=pn*multicoeff(k); endif; endfor; endfor; 
PA
PA =  0.91943
octave:85> n=100; PA=0; count=0; for k1=0:n, for k2=0:n-k1, 
k=[k1, k2, n-k1-k2]; lpn=-(1/n)*(sum(k.*log2(p))); pn=p(1)^k1*p(2)^k2*p(3)^(n-k1-k2); 
if abs(lpn-H)<eps, count++; PA+=pn*multicoeff(k); endif; endfor; endfor; 
PA
PA =  0.78298
octave:86> n=10; PA=0; count=0; for k1=0:n, for k2=0:n-k1, 
k=[k1, k2, n-k1-k2]; lpn=-(1/n)*(sum(k.*log2(p))); pn=p(1)^k1*p(2)^k2*p(3)^(n-k1-k2); 
if abs(lpn-H)<eps, count++; PA+=pn*multicoeff(k); endif; endfor; endfor; 
PA
PA =  0.21106
\end{lstlisting}

\part 
to-do

\end{parts}
\end{solution}
\end{questions}
