\subsection{Tipicidade nas provas}
% ~/ee/ufsj/2017_02/ti/provas/prova01-resolucao.tex

\begin{questions}
\question{
Suponha que uma turma com 30 alunos realizou uma prova com 7 questões de múltipla escolha
onde cada questão possuía 7 alternativas. Suponha que os alunos não tenham estudado e, sem 
saber resolver as questões, resolveram marcar aleatoriamente (sem preferência por nenhuma opção
e sem nenhum critério para escolher as opções) as respostas

\begin{parts}
\part
Calcule a probabilidade de que todos  os alunos tenham acertado no mínimo 4 questões.

\part
Foi observado que os alunos, sem apresentar a resolução de nenhuma das questões, 
acertaram no mínimo 4 questões. Utilizando os seu conhecimentos da Propriedade da Equipartição 
Assintótica, podemos concluir que esta observação pertence ou não ao conjunto típico? Por que?
(considere apenas uma aproximação, uma vez que o conjunto típico surge quando observamos
sequências arbitrariamente longas, ou seja, é um comportamento assintótico)

\part
Podemos concluir que este grupo de aluno utilizou-se de algum meio ilícito para obter as
respostas das questões? Qual é o grau de confiabilidade desta conclusão?

\part
Explique o que é o conjunto típico e quais resultados seriam resultados típicos para esta prova,
considerando que os alunos escolheram aleatoriamente as respostas.
(Não é necessário realizar as contas, apenas deixe indicado como resolver a questão).

\end{parts}
}

\begin{solution}
Em cada questão o aluno pode acertar ou errar, então podemos ver o resultado
de cada questão como uma realização de um experimento de Bernoulli com $p=1/7$,
uma vez que as opções de cada questão possuem a mesma probabilidade de serem escolhidas,
ou seja, teremos uma probabilidade de $1/7$ para o acerto e $6/7$ para o erro.
Um resultado dos 30 exames pode ser visto como uma sequência de comprimento $N=30 \times 7 = 210$
(30 alunos e 7 questões por prova) de v.a. i.i.d., $X_{1:N}$, onde $X_i \sim \text{Bern}(p)$, $\forall i$,
ou então como uma realização de uma v.a. Binomial, $X \sim \text{Bin}(N,k)$, onde $N$ é
o número de experimentos de Bernoulli e $k$ o número de sucessos.

\begin{parts}
\part
Para calcular a probabilidade de que todos os alunos tenham acertado no mínimo
4 questões a esmo, devemos somar a probabilidade de todas as observações que satisfazem
esta condição, ou seja, todas as sequências de $N$ realizações do ensaio de Bernoulli
onde ocorram $30 \times 4 = 120$ ou mais sucessos, ou seja, a probabilidade da v.a.
binomial assumir um valor igual ou superior a $120$.

\begin{equation}
\Pr(\textmd{\# acertos} \geq 4) = \Pr(X \geq 120) = \sum_{k=120}^{N} {N \choose k} p^{k}(1-p)^{N-k} = 4.4\times 10^{-47} .
\end{equation}

\part
Esta observação não pertence ao conjunto típico. As sequências típicas são aquelas
que apresentam aproximadamente $N \times p = 30$ sucessos e $N \times (1-p) = 180$ falhas.
Como a variância de uma v.a. Binomial é $\Var(X) = Np(1-p) = 180/7 \approx 25,7$, então
o desvio padrão será $\sigma \approx 5$. Para uma distribuição Gaussiana, sabemos que
ao tomar 3 desvios padrões acima e abaixo da média nos confere 99\% de probabilidade.
Para o caso Binomial não é muito diferente. Esperamos então que, ao tomar as
os casos com 15 a 45 sucessos ($30 - 3 \times 5$ e $30 + 3 \times 5$) teremos aproximadamente
toda a probabilidade. Desta forma, podemos considerar como observações típicas aquelas em que observamos
de 15 a 45 sucessos. Uma observação com 120 sucessos ou mais está bem distante do que é típico.

De forma mais rigorsa, devemos definir $\epsilon$, por exemplo $\epsilon = 0.1$.
Dado este valor de $\epsilon$ as sequências típicas serão aquelas que satisfazem
\begin{equation}
H(X)-\epsilon \leq -\frac{1}{n} \log p(x_{1:n}) \leq H+\epsilon .
\end{equation}
Utilizando o seguinte código podemos determinar quais são estas sequências:
\begin{verbatim}
>> n=210; p=1/7; eps=0.1;
>> H=-p*log2(p)-(1-p)*log2(1-p);
>> for k=0:n, 
       pn=p^k*(1-p)^(n-k); 
       lpn=-(1/n)*log2(pn); 
       if lpn>H-eps && lpn<H+eps, disp(k);endif; 
   endfor;
\end{verbatim}
e assim concluímos que as as sequências com $k \in [22,38]$ pertencem ao conjunto típico.


\part
Sim. Não é factível observar tantos sucessos. Podemos afirmar com praticamente 100\%
de certeza que algo ilícito ocorreu (considerando os valores de .

Se considerarmos o conjunto típico $A_{\epsilon}^{(n)}$
como o conjunto das sequências com 5 a 45 sucessos, teremos
\begin{equation}
\Pr(A_{\epsilon}^{(n)}) = \sum_{k=15}^{45} {N \choose k} p^{k}(1-p)^{N-k} = 0.99768 ,
\end{equation}
ou seja, a probabilidade do conjunto não-típcio é $0.002$. Se usarmos 4 desvios padrões,
ao invés de 3, como feito anteriormente, passaremos a ter $\Pr(A_{\epsilon}^{(n)}) = 0.99991$,
e assim a probabilidade do conjunto não-típico será $0.00009$.

Realizando os cálculos com $\epsilon=0.1$, teremos
\begin{equation}
\Pr(A_{\epsilon}^{(n)}) = \sum_{k=22}^{38} {N \choose k} p^{k}(1-p)^{N-k} = 0.90735 ,
\end{equation}
ou seja, podemos confirmar que $\Pr(A_{\epsilon}^{(n)}) > 1 - \epsilon$.

\begin{verbatim}
>> n=210; p=1/7; eps=0.1; PA=0;
>> H=-p*log2(p)-(1-p)*log2(1-p);
>> for k=0:n, 
        pn=p^k*(1-p)^(n-k); 
        lpn=-(1/n)*log2(pn); 
        if lpn>H-eps && lpn<H+eps,
           PA+=nchoosek(n,k)*pn; 
        endif; 
   endfor;
\end{verbatim}


\part
Conjunto típico é o menor conjunto que detém `toda' a probabilidade. As sequências típicas
são aquelas que possuem probabilidade de $2^{-nH}$, onde $n$ é o comprimento da sequência
e $H$ a entropia da fonte. No caso em questão, seriam resultados típicos observações
com aproximadamente 30 sucessos. Como mostrado acima, considerando as sequências com
15 a 45 sucessos, este conjunto deterá mais de 99\% da probabilidade.
Se utilizarmos o $\epsilon = 0.1$, consideraremos as sequências com 22 a 38 sucessos com típicas,
resultando em um conjunto com probabilidade maior que 90\%.

A título de informação, para ganhar na megasena é necessário acertar um jogo de 6 números
escolhido dentre os números de 1 a 60. O número de possíveis jogos na megasena é
\begin{equation}
{60 \choose 6} = 50063860
\end{equation}
Como todos os jogos são equiprováveis, a chance de ganhar na megasena com um jogo é
de 1 em 50063860, ou seja, $1.99 \times 10^{-8}$.
A probabilidade de observarmos 4 ou mais acertos nas questões dos 30 alunos é menor
do que a probabilidade de ganhar 6 vezes na megasena.

\begin{verbatim}
>> Pmegasena = 1/50063860;
>> P=0; 
>> for k=120:n, 
       pn=p^k*(1-p)^(n-k); 
       P+=nchoosek(n,k)*pn; 
   endfor;
>> log(P)/log(Pmegasena)
ans =  6.0202
\end{verbatim}


\end{parts}
\end{solution}
\end{questions}
