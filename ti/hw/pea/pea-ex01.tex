\subsection{Propriedade da Equipartição Assintótica e Codificação de Fonte} 

\begin{questions}
\question{
Uma fonte discreta sem memória emite uma sequência de dígitos binários estatisticamente independentes 
com probabilidades $p(1)=0.005$ e $p(0)=0.995$. Os dígitos são tomados $100$ a cada vez e uma palavra 
de código (\emph{codeword}) binária é fornecido para cada sequência de $100$ dígitos contendo três ou menos uns.

\begin{parts}
\part
Assumindo que todas as palavras de código possuem o mesmo comprimento, encontre o menor comprimento 
necessário para fornecer palavras de código para todas as sequências com três ou menos uns.

\part 
Calcule a probabilidade de se observar uma sequência produzida pela fonte para a qual não foi associada nenhuma palavra de código.

\part 
Utilize a desigualdade de Chebyshev para encontrar o limite de se observar uma sequência da fonte para 
a qual nenhuma palavra código foi associada. Compare este limite com a probabilidade calculada no item anterior.
\end{parts}
}

\begin{solution}
\begin{parts}
\part
O número de sequências binárias com 3 ou menos uns é
\begin{equation}
{100 \choose 0} + {100 \choose 1} + {100 \choose 2} + {100 \choose 3} = 1 + 100 + 4950 + 161700 = 166751 .
\end{equation}
Para codificar por uma palavra binárias as sequências de comprimento 100 com 3 ou menos uns,
serão necessários $\lceil \log 166751 \rceil = 18$ bits.
Note que $H(0.005) = 0.0454$ e $18$ bits é consideravelmente maior que $4.5$ bits de entropia
para sequências de tamanho $100$.


\part
A probabilidade de se observar uma sequência produzida pela fonte para a qual não foi
associada nenhuma palavra de código é equivalente à soma da probabilidade das sequências
com mais de 3 uns, ou então, um menos a probabilidade das sequências com 3 ou menos uns.
\begin{eqnarray}
\sum_{i=4}^{100} {100 \choose i} (0.005)^i (0.995)^{100-i} &=& 1 - \sum_{i=0}^{3} {100 \choose i} (0.005)^i (0.995)^{100-i} \nonumber \\
        &=& 1 - (0.60577 + 0.30441 + 0.7572 + 0.01243) \nonumber \\
	&=& 1 - 0.99833 = 0.00167 .
\end{eqnarray}


\part 
No caso em que uma v.a. $S_n$ é a soma de $n$ v.a.s i.i.d. $X_1, \ldots, X_n$,
a desigualdade de Chebyshev afirma que
\begin{equation}
\Pr ( |S_n - n \mu| \geq \epsilon ) \leq \frac{n \sigma^2}{\epsilon^2} ,
\end{equation}
onde $\mu$ e $\sigma^2$ são a média e variância de $X_i$, respectivamente.
(Desta forma, $n\mu$ e $n\sigma^2$ são a média e variância de $S_n$).
No problema em questão temos: $n=100$, $\mu=0.005$ e $\sigma^2 = (0.005)(0.995)$.
As sequência que não serão codificadas são aquelas em que $S_n \geq 4$.
Note que teremos $S_n \geq 4$ se e somente se $|S_{100} - 100 (0.005)| \geq 3.5$.
Devemos pois escolher $\epsilon = 3.5$, assim
\begin{equation}
\Pr(S_{100} \geq 4) \leq \frac{100 (0.005)(0.995)}{(3.5)^2} \approx 0.04061 .
\end{equation}
Verificamos assim que o limite é bem maior que o valor da probabilidade de $0.00167$.

\end{parts}
\end{solution}
\end{questions}
