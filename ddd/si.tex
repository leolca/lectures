\begin{frame}[allowframebreaks,fragile]
\frametitle{siunitx}

O pacote \texttt{siunitx} é utilizado para formatação de texto com quantidades físicas de maneira consistente no texto.

\begin{LTXexample}
\num{12345,67890}  \\
\num{.3e45}        \\
\num{0.123}   \\
\num{0,1234}  \\
\num{3.45d-4} \\
\num{3.4567e-6} 
\end{LTXexample}

\framebreak

\begin{LTXexample}
\num{1.23456}     \\
\num{14.23}      \\
\sisetup{round-mode = places, round-precision = 3}%
\num{1.23456}     \\
\num{14.23}      \\
\end{LTXexample}

\framebreak

\begin{LTXexample}
\begin{table}\caption{Uso tabela sem o pacote \texttt{siunitx}.}
\begin{tabular}{l}
\toprule{Some Values} \\
\midrule 
2.3456 \\ 34.2345 \\ -6.7835 \\ 90.473  \\ 5642.5    \\ 1.2e3 \\ e4  \\
\bottomrule
\end{tabular}
\end{table}
\end{LTXexample}

\framebreak 

\begin{LTXexample}
\begin{table}\caption{Standard behaviour of the \texttt{S} column type.%
\label{tab-S-standard}}
\begin{tabular}{@{}S@{}}
\toprule{Some Values} \\
\midrule 
2.3456 \\ 34.2345 \\ -6.7835 \\ 90.473  \\ 5642.5    \\ 1.2e3 \\ e4  \\
\bottomrule
\end{tabular}
\end{table}
\end{LTXexample}



\end{frame}
