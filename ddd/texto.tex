\section{Texto}
\begin{frame}
\frametitle{Estrutura}
\onslide<1->{Um manuscrito deve possuir uma estrutura rígida e bem definida.}
\begin{itemize}
\item<2-> organização
\item<3-> comunicar de forma clara a mensagem
\item<4-> eliminar redundâncias desnecessárias
\end{itemize}
\onslide<5->{Mesmo que seja um assunto técnico, há espaço para criatividade e elaborar uma narrativa instigadora e convincente.}
\end{frame}

\begin{frame}
\frametitle{Defina o seu público}
O leitor quer rapidamente compreender os conceitos e conclusões.

Ao mesmo tempo, o escritor deseja mostrar a importância de sua contribuição e
convencer outros especialistas da área.
\end{frame}

\begin{frame}
\frametitle{Título e resumo}
O título é a porta de entrada e deve comunicar a principal contribuição do trabalho.

\vspace{3em}
O resumo deve contar a mensagem geral do trabalho, evidenciando o principal resultado.
\end{frame}

\begin{frame}
\frametitle{Sentenças}
Utilize sentenças curtas.
\end{frame}

\begin{frame}
\small 
Não imite Proust. Não utilize períodos longos. Se vos acontecer fazê-los, 
dividam-nos depois. Não receiem repetir duas vezes o sujeito. 
Eliminem o excesso de pronomes e de orações subordinadas. Não escrevam:
\begin{quote}
O pianista Wittgenstein, que era irmão do conhecido filósofo que escreveu 
o \emph{Tractatus Logico-Philosophicus} que hoje em dia muitos consideram 
a obra-prima da filosofia contemporânea, teve a ventura de Ravel ter escrito 
para ele o concerto para a mão esquerda, dado que tinha perdido a direita na 
guerra.
\end{quote}
Mas escrevam, quando muito:
\begin{quote}
O pianista Paul Wittgenstein era irmão do filosofo Ludwig Wittgenstein. 
Como Paul era mutilado da mão direita, o compositor Maurice Ravel escreveu 
para ele o concerto para a mão esquerda.
\end{quote}
\begin{flushright}
Umberto Eco, \emph{Como se faz uma tese em Ciências Humanas} (1977).
\end{flushright}
\end{frame}
\note{
\tiny
"Sem honra, senão precária; sem liberdade, senão provisória, até a descoberta do crime; sem posição que não seja instável, como para o poeta, festejado na véspera em todos os salões, aplaudido em todos os teatros de Londres e, no dia seguinte, expulso de todos os quartos, sem poder achar um travesseiro onde repousar a cabeça, dando voltas à pedra de amolar como no verso do Poema " A cólera de Sansão", de Alfred de Vigny (1797-1863) como Sansão, ele fica repetindo: ''Os dois sexos morrerão cada qual por seu lado; excluídos até, salvo nos dias de grande infelicidade, em que a maioria se reúne ao redor de sua vítima”; como os judeus ao redor de Dreyfus de toda simpatia, e às vezes da sociedade, de seus semelhantes, aos quais dão o desgosto de ver que são, pintados num espelho que, não os adulando mais, acusa todas as taras que não tinham desejado notar em si mesmos e que os faz compreenderem que aquilo a que denominam amor (e a que, brincando com a palavra, haviam anexado, por sentido social, tudo quanto a poesia, a pintura, a música, a cavalaria, o ascetismo tinham podido acrescentar ao amor) decorre não de um ideal de beleza que tenham escolhido, mas de uma enfermidade incurável; como ainda os judeus (salvo uns poucos que só desejam conviver com os de sua raça, e têm sempre nos lábios as palavras rituais e os gracejos consagrados), fugindo uns dos outros, buscando os que lhes são mais contrários, que não querem saber deles, perdoando as suas zombarias, embriagando-se com suas complacências; mas ainda assim unidos a seus semelhantes pelo ostracismo que os fere, o opróbrio em que caíram, tendo acabado por adquirir, graças a uma perseguição idêntica à de Israel, os caracteres físicos e morais de uma raça, às vezes bela, freqüentemente horrível, encontrando apesar de todas as troças com que o mais mesclado, mais assimilado à raça adversa, é relativamente, em aparência, o menos invertido, cobre aquele que simplesmente continuou a sê-lo um descanso no convívio de seus semelhantes, e até um apoio na existência, até que, negando sempre formarem uma raça (cujo nome é a maior injúria), os que conseguem ocultar que a ela pertencem, desmascaram-nos de boa vontade, não tanto para lhes causar dano, coisa que não detestam, quanto para se desculparem, e indo buscar, como um médico pesquisa o apendicite, a inversão até na História, tendo prazer em lembrar que Sócrates era um deles, como os israelitas dizem que, era judeu, sem pensar que não havia anormais quando o homossexual a regra, nem anticristãos antes de Jesus Cristo, que só o opróbrio no crime, pois só deixou de subsistir para aqueles que eram refratários de toda pregação, a todo exemplo, a todo castigo, em virtude de uma distinção inata e de tal modo especial que repugna mais aos outros homens daquele que possa vir acompanhado de altas qualidades morais) do que vícios que se contradizem, como o roubo, a crueldade, a má-fé, mais compreendidos e, portanto, mais desculpados pelo comum dos homens; - formando uma franco-maçonaria bem mais extensa, mais eficaz e suspeita que a das lojas, pois repousa numa identidade de gostos, aparências, de hábitos, de perigos, de aprendizagem, de saber, de tráfico; glossários, e na qual os próprios membros que aspiram a não ser conhecidos logo se reconhecem por traços naturais ou de convenção, involuntárias ou intencionais, que assinalam ao mendigo um de seus semelhantes o grão-senhor que lhe fecha a porta de seu carro; ao pai, no noivo da filha; ao que desejara curar-se, confessar-se, defender-se, no médico, no pai, no advogado a quem recorreu; todos forçados a proteger o seu segredo; tendo a sua parte no segredo dos outros, de que o restante da humanidade - não suspeita e que faz com que os mais inverossímeis romances de aventuras lhes pareçam verdadeiros; pois, nessa vida romanesca, anacrônica; o embaixador é amigo do preso; o príncipe, com uma certa liberdade dos que lhe confere a educação aristocrática e que um pequeno-burguês medroso não teria, ao sair da casa da duquesa, vai se entender como apache; parte reprovada da coletividade humana, porém parte importante, que se suspeita onde não está, ostensiva, insolente, impune onde é adivinhada; contando com adeptos por toda a parte, no povo, no exército, no templo, na penitenciária, no trono; vivendo enfim, ao menos um grande número, na intimidade cariciosa e arriscada dos homens da outra parte, provocando-os, brincando com eles ao falar do seu vício como se não fosse seu; jogo que se torna fácil pela cegueira ou pela falsidade dos outros, já que pode se prolongar durante anos até o dia do escândalo, em que domadores são devorados; até então obrigados a ocultar a sua vida, a virar os olhos de onde gostariam de fixá-los, a fixá-los de onde gostariam de desviá-los, de mudar o gênero de muitos adjetivos em seu vocabulário; o freio social em comparação com o freio interior que seu vício, ou o que denomina impropriamente desse modo, lhes impõe não mais em relação à outros mas a si mesmos, e de maneira que a eles próprios não pareça vício."

Sodoma e Gomorra (Marcel Proust)
}





\begin{frame}
\frametitle{Fluxo de ideias}

\begin{itemize}
\item Estabeleça um fluxo de ideias. 
\item Evite zig-zag.
\item Utilize paralelismos. 
\item Agrupe as ideias.
\end{itemize}

% For example, if we have three independent reasons why we prefer one 
% interpretation of a result over another, it is helpful to communicate 
% them with the same syntax so that this syntax becomes transparent to 
% the reader, which allows them to focus on the content.
% There is nothing wrong with using the same word multiple times in a sentence or paragraph.
% Resist the temptation to use a different word to refer to the same concept—doing so makes 
% readers wonder if the second word has a slightly different meaning.
\end{frame}
\note{
Paralelismos

\vspace{3ex}
Suponha que você queira comunicar diferentes resultados para um experimento ou
diferentes explicações para uma determinada observação. O paralelismo é útil nestes
casos. Busque utilizar a mesma sintaxe para descrever cada um dos resultados ou para
tecer cada umas das explicações. Desta forma a sintaxe fica transparente e o leitor 
pode focar no conteúdo. Neste caso, não há problemas em repetir palavras em uma 
sentença ou parágrafo. Resista à tentação de utilizar palavras diferentes para 
referir-se a um mesmo conceito, evitando assim que o leitor tenha dúvidas quanto
à equivalência das mesmas.
}




\begin{frame}
\frametitle{Empatia}
Coloque-se no lugar do leitor. Obtenha \emph{feedback} de terceiros.

\vspace{3ex}
A escrita é um processo de otimização.

\vspace{3ex}
Muitas vezes é necessário desapego.
% it is important not to get too attached to one’s writing. 
% In many cases, trashing entire paragraphs and rewriting is a faster way to produce good text than incremental editing.
\end{frame}


% Why does scientific writing have to be stodgy, dry and abstract? Humans are story-telling animals. If we don’t engage that aspect of ourselves, it’s hard to absorb the meaning of what we’re reading. Scientific writing should be factual, concise and evidence-based, but that doesn’t mean it can’t also be creative — told in a voice that is original — and engaging (Z. A. Doubleday et al. Trends Ecol. Evol. 32, 803–805; 2017). If science isn’t read, it doesn’t exist.
% https://www.nature.com/articles/d41586-018-02404-4

% -------------------




\begin{frame}
\small
Sugestões de leitura: 
\vspace{2ex}

\fullcite{gewin_how_2018}
\vspace{2ex}

\fullcite{mensh_ten_2017}
\vspace{2ex}

\fullcite{smith_task_1990}
\vspace{2ex}

\fullcite{eco_how_2015}

\end{frame}
