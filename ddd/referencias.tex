\subsection{Rótulos e referências}

\begin{frame}[fragile]
\frametitle{Rótulos e referências internas}
Em um texto em \LaTeX{} é possível referenciar quase tudo que é numerado em um documento.
Por exemplo: figuras, tabelas, listas, páginas, secções, capítulos, equações, notas de rodapé, etc.
\pause

\vspace{3ex}
\begin{itemize}[<-+>]
\item \verb|\label{rotulo}|: fornecer um rótulo ao objeto que se deseja referenciar
\item \verb|\ref{rotulo}|: realizar a referencia ao objeto com um dado rótulo
\item \verb|\pageref{rotulo}|: referenciar a página onde o objeto se encontra
\end{itemize}

\end{frame}


\begin{frame}[fragile]
\frametitle{Rótulos e referências internas}
\begin{LTXexample}
O teorema de Pit\'agoras \'e equacionado como
\begin{equation}
\label{eq-pitagoras}
c^2 = a^2 + b^2
\end{equation}

...

Veja a Equa\c{c}\~ao~\ref{eq-pitagoras}.
\end{LTXexample}
\end{frame}

