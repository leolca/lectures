\section{Automatização}
\begin{frame}
\frametitle{Automatização}
Automatização é util para realizar tarefas repetitivas e com grande volume de informação.

\vspace{3ex}
\textbf{Bash} (Bourne Again Shell) é um shell do Unix (um interpretador de linhas de comando).

\vspace{3ex}
\fullcite{yitbarek2019}
\end{frame}


\subsection{Bash}
\begin{frame}[allowframebreaks,fragile]
\frametitle{Bash}
\texttt{Bash} foi escrito por Brian Fox para o projeto GNU. A primeira versão saiu em 1989, tornando-se o shell padrão 
para a maioria das distribuições Linux e para macOS (até 2019, quando a Apple passou a adotar o \texttt{zsh}).

\vspace{2ex}
O nome \texttt{bash} é um acrônimo para \emph{Bourne Again Shell}.


\vspace{2ex}
Windows 10: \href{https://blogs.windows.com/buildingapps/2016/03/30/run-bash-on-ubuntu-on-windows/}{Windows Subsystem For Linux},
\href{https://www.howtogeek.com/249966/how-to-install-and-use-the-linux-bash-shell-on-windows-10/}{How to Install and Use the Linux Bash Shell on Windows 10}

Windows: \href{https://www.cygwin.com/}{Cygwin}


\href{https://www.virtualbox.org/}{VM VirtualBox}

Online: 
\href{https://colab.research.google.com}{Google Colab},
\href{https://cocalc.com}{Cocalc},
\href{https://jupyter.org/}{Jupyter Notebook}.

\vspace{2ex}
\url{https://en.wikipedia.org/wiki/Bash_(Unix_shell)}


\framebreak

\begin{lstlisting}[language=bash, label=lst-bash-new-script-01, caption={Criando um script.}, postbreak=\mbox{$\hookrightarrow$\space}, basicstyle=\fontsize{8}{10}\selectfont\ttfamily]
$ touch myscript.sh
$ chmod +x myscript.sh
\end{lstlisting}

\begin{lstlisting}[language=bash, label=lst-bash-new-script-02, caption={Criando um script.}, postbreak=\mbox{$\hookrightarrow$\space}, basicstyle=\fontsize{8}{10}\selectfont\ttfamily]
#!/bin/bash
printf "Hello %s!\n" "$1"
\end{lstlisting}

\begin{lstlisting}[language=bash, label=lst-bash-convert-imgs, caption={Script para converter PNGs em JPGs}, postbreak=\mbox{$\hookrightarrow$\space}, basicstyle=\fontsize{8}{10}\selectfont\ttfamily]
for img in $( ls *.png ); do convert $img  ${img%%png}jpg; done
\end{lstlisting}

\begin{lstlisting}[language=bash, label=lst-bash-sequencia, caption={Sequência de 1 a 100.}, postbreak=\mbox{$\hookrightarrow$\space}, basicstyle=\fontsize{8}{10}\selectfont\ttfamily]
for i in `seq 100`; do echo $i; sleep 1; done
\end{lstlisting}

\begin{lstlisting}[language=bash, label=lst-bash-listpdfs, caption={Listar PDFs em ordem decrescente de tamanho}, postbreak=\mbox{$\hookrightarrow$\space}, basicstyle=\fontsize{8}{10}\selectfont\ttfamily]
ls -la *.[pP][dD][fF] | tr -s [:blank:] | cut -d' ' -f5,9 | column -t | sort -nr > pdflist.txt
\end{lstlisting}

Lista de alguns comandos uteis:
\begin{description}
\item[ls] listar conteúdo de diretórios
\item[echo] imprimir texto na tela
\item[touch] criar arquivo
\item[mkdir] criar diretório
\item[grep] procurar textos e padrões
\item[man] manual, obter ajuda para os comandos
\item[pwd] diretório corrente
\item[cd] trocar de diretório
\item[mv] mover arquivo
\item[rm] remover arquivo
\item[rmdir] remover diretório
\item[cp] copiar aquivos
\item[less] visualizar o conteúdo de arquivos
\item[cat] concatenar e imprimir na saída padrão
\item[>, <, |] redirecionamentos
\item[head] lê o início de um arquivo
\item[tail] lê o final de um arquivo
\item[chmod] alterar permissões
\item[kill] matar um processo
\item[sleep] esperar um determinado tempo
\item[\&] executar em background
\item[cal] calendário
\item[diff] verificar diferenças entre arquivos
\item[paste] juntar linhas de arquivos
\item[ps] status de processos
\item[sort] ordenar
\item[tr] traduzir, comprimir e/ou remover caracteres
\item[sed] editor de fluxo
\item[awk] linguagem para processamento de texto
\end{description}

\framebreak

\begin{lstlisting}[language=bash, label=lst-bash-simple-loop, caption={Criando um loop simples.}, postbreak=\mbox{$\hookrightarrow$\space}, basicstyle=\fontsize{8}{10}\selectfont\ttfamily]
#!/bin/bash
for i in 1 2 3 4 5; do
  echo "counter: $i"
done

for i in {1..10}; do  echo "counter: $i"; done

for counter in $(seq 1 255); do echo "$counter"; done
\end{lstlisting}

\framebreak

\begin{itemize}
\item entrada padrão (\emph{stdin})
\item saída padrão (\emph{stdout})
\item saída de erro padrão (\emph{stderr})
\item redirecionamentos e \emph{pipes}
\item descritores de arquivos
\end{itemize}

\begin{lstlisting}[language=bash, label=lst-bash-stds, caption={Entrada e saída padrão.}, postbreak=\mbox{$\hookrightarrow$\space}, basicstyle=\fontsize{8}{10}\selectfont\ttfamily]
# envia uma msg para saída padrão
echo hello

# ler da entrada padrão e escreve na saída padrão (até encontrar EOF (ctrl+d))
cat
\end{lstlisting}

\begin{lstlisting}[language=bash, label=lst-bash-pipe, caption={Pipes.}, postbreak=\mbox{$\hookrightarrow$\space}, basicstyle=\fontsize{8}{10}\selectfont\ttfamily]
# um pipe conecta a saída padrão de um comando com a entrada padrão do próximo
echo "hello there" | sed "s/hello/hi/"

echo "hello there" | sed "s/hello/hi/" | sed "s/there/robots/"
\end{lstlisting}

\begin{lstlisting}[language=bash, label=lst-bash-stderr, caption={Erros.}, postbreak=\mbox{$\hookrightarrow$\space}, basicstyle=\fontsize{8}{10}\selectfont\ttfamily]
cat arquivo 

cat arquivo | sed 's/file/arquivo/'
\end{lstlisting}


\begin{lstlisting}[language=bash, label=lst-bash-redirecionamento, caption={Redirecionamento.}, postbreak=\mbox{$\hookrightarrow$\space}, basicstyle=\fontsize{8}{10}\selectfont\ttfamily]
echo hello > arquivo
cat arquivo

echo hello again >> arquivo
cat arquivo
\end{lstlisting}

\framebreak
Descritor de arquivo (FD, \emph{file descriptor}) é um número inteiro que refere a uma fonte de entra/saída.
Exemplos: 0 (\emph{stdin}), 1 (\emph{stdout}) e 2 (\emph{stderr}). Para redirecionar um descritor utizamos o operador \verb|>&|.
\begin{lstlisting}[language=bash, label=lst-bash-descritores, caption={Descritores de arquivos.}, postbreak=\mbox{$\hookrightarrow$\space}, basicstyle=\fontsize{8}{10}\selectfont\ttfamily]
# redireciona o stdout para stderr
echo "hello there" >&2
# equivalente a fazer
echo "hello there" 1>&2
# onde explicitamos que o FD 1 (stdout) será redirecionado para FD 2 (stderr)

# agora um exemplo redirecionando a saída do sterr em na saída stdout
cat arquivo 2>&1 | sed 's/file/arquivo/'

# redirecionando para arquivos de log (stdout para stdout.log e stderr para debug.log)
python hello.py 1>stdout.log 2>debug.log

# redirecionando stdout e stderr para um único arquivo de log
python hello.py 1>all_output.log 2>&1

# redirecionando um arquivo para a entrada padrão de um programa em python
python3 hello.py < input.txt
cat input.txt | python3 hello.py

# redirecionando entrada e saída
python3 hello.py < input.txt > output.txt
\end{lstlisting}




\framebreak

\begin{lstlisting}[language=bash, label=lst-bash-ping-loop, caption={Tempo de ping nas máquinas.}, postbreak=\mbox{$\hookrightarrow$\space}, basicstyle=\fontsize{8}{10}\selectfont\ttfamily]
for counter in {1..255}
do 
  ptime=$(ping -c 1 192.168.0.$counter | grep -oP "(?<=time=)([0-9].[0-9]+)")
  echo -e "192.168.0.$counter\t${ptime:-0}"
done
\end{lstlisting}

\framebreak

\begin{lstlisting}[language=bash, label=lst-bash-if, caption={if else em Bash.}, postbreak=\mbox{$\hookrightarrow$\space}, basicstyle=\fontsize{8}{10}\selectfont\ttfamily]
#!/bin/bash

VAR=$1
if [ -z "$VAR" ]
then
  echo -n "Enter a number: "
  read VAR
fi

if [[ $VAR -gt 10 ]]
then
  echo "The variable is greater than 10."
else
  echo "The variable is equal or less than 10."
fi
\end{lstlisting}

\framebreak

\begin{lstlisting}[language=bash, label=lst-argumentos, caption={argumentos Bash.}, postbreak=\mbox{$\hookrightarrow$\space}, basicstyle=\fontsize{8}{10}\selectfont\ttfamily]
#!/bin/bash

echo "Início do programa: $(date)"
echo "Nome do programa: $0"
echo "Número de argumentos $#"
echo "Número do processo $$"

COUNTER=0
for arg in "$@"
do
  ((COUNTER++))
  echo "arg $COUNTER: $arg"
done
\end{lstlisting}

\framebreak


\begin{lstlisting}[language=bash, label=lst-bash-convert-mp3, caption={Converter todos WAVs em MP3.}, postbreak=\mbox{$\hookrightarrow$\space}, basicstyle=\fontsize{8}{10}\selectfont\ttfamily]
for file in $( ls *.wav ); do lame $file ${file%wav}mp3; done

for file in $( ls *.wav ); do lame -m j -h --vbr-new -b 160 -s 44.1 $file -o ${file%wav}mp3; done
\end{lstlisting}

\begin{lstlisting}[language=bash, label=lst-bash-thumbnails, caption={Criar minuaturas de todas fotos.}, postbreak=\mbox{$\hookrightarrow$\space}, basicstyle=\fontsize{8}{10}\selectfont\ttfamily]
for img in $(ls *.jpg); do echo $img; convert $img -resize 75x75 ${img%%.jpg}_thumbnail.jpg; done
\end{lstlisting}

\framebreak

\begin{lstlisting}[language=bash, label=lst-bash-word-freq, caption={Listando as palavras por frequência de ocorrência.}, postbreak=\mbox{$\hookrightarrow$\space}, basicstyle=\fontsize{8}{10}\selectfont\ttfamily]
cat ulysses.txt | tr 'A-Z' 'a-z' | tr -c "a-z'" "\n" | sed '/^$/d' | sort | uniq -c | sort -k1nr -k2 > wordlist.txt

cat ulysses.txt | tr 'A-Z' 'a-z' | tr -c "a-z'" "\n" | sed '/^$/d' | sort | uniq -c | sort -k1nr -k2 | tail -n 100 | pr -c5 -t -w80
\end{lstlisting}

\framebreak

\begin{lstlisting}[language=bash, label=lst-bash-misspelled, caption={Listando as palavras não encontradas no dicionário. Salvar no arquivo \texttt{misspelled.sh}.}, postbreak=\mbox{$\hookrightarrow$\space}, basicstyle=\fontsize{8}{10}\selectfont\ttfamily]
#!/bin/bash

comm -13 <(cat /usr/share/dict/american-english | tr 'A-Z' 'a-z' | sort | uniq) <(cat "${1:-/dev/stdin}" | tr 'A-Z' 'a-z' | tr -c "a-z'" "\n" | sed '/^$/d' | sort | uniq)
\end{lstlisting}

\framebreak

\begin{lstlisting}[language=bash, label=lst-bash-misspelled2, caption={Listando as palavras de Ulysses não encontradas no dicionário.}, postbreak=\mbox{$\hookrightarrow$\space}, basicstyle=\fontsize{8}{10}\selectfont\ttfamily]
chmod +x misspelled.sh
cat ulysses.txt | ./misspelled.sh 
cat ulysses.txt | ./misspelled.sh | wc -l
./misspelled.sh ulysses.txt | wc -l

# listar palavras, removendo aquelas com apostrofe, buscar a frequencia de ocorrencia e ordenar
./misspelled.sh ulysses.txt | grep -v "'" | while read word; do grep -P "\s$word$" wordlist.txt; done | sort -k1nr -k2
\end{lstlisting}

\framebreak

\fullcite{evans2021}
\vspace{2ex}

\fullcite{devhints}
\vspace{2ex}

\fullcite{robbins2016}
\vspace{2ex}

\fullcite{robbins_classic_2005}

\end{frame}


