\section{Automatização}
\begin{frame}
\frametitle{Automatização}
Automatização é util para realizar tarefas repetitivas e com grande volume de informação.

\vspace{3ex}
\textbf{Bash} (Bourne Again Shell) é um shell do Unix (um interpretador de linhas de comando).

\vspace{3ex}
\fullcite{yitbarek2019}
\end{frame}


\subsection{Bash}
\begin{frame}[allowframebreaks,fragile]
\frametitle{Bash}
\texttt{Bash} foi escrito por Brian Fox para o projeto GNU. A primeira versão saiu em 1989, tornando-se o shell padrão 
para a maioria das distribuições Linux e para macOS (até 2019, quando a Apple passou a adotar o \texttt{zsh}).

\vspace{2ex}
O nome \texttt{bash} é um acrônimo para \emph{Bourne Again Shell}.


\vspace{2ex}
Windows 10: \href{https://blogs.windows.com/buildingapps/2016/03/30/run-bash-on-ubuntu-on-windows/}{Windows Subsystem For Linux},
\href{https://www.howtogeek.com/249966/how-to-install-and-use-the-linux-bash-shell-on-windows-10/}{How to Install and Use the Linux Bash Shell on Windows 10}

Windows: \href{https://www.cygwin.com/}{Cygwin}


\href{https://www.virtualbox.org/}{VM VirtualBox}

Online: 
\href{https://colab.research.google.com}{Google Colab},
\href{https://cocalc.com}{Cocalc},
\href{https://jupyter.org/}{Jupyter Notebook}.

\vspace{2ex}
\url{https://en.wikipedia.org/wiki/Bash_(Unix_shell)}


\framebreak

\begin{lstlisting}[language=bash, label=lst-bash-new-script-01, caption={Criando um script.}, postbreak=\mbox{$\hookrightarrow$\space}, basicstyle=\fontsize{8}{10}\selectfont\ttfamily]
$ touch myscript.sh
$ chmod +x myscript.sh
\end{lstlisting}

\begin{lstlisting}[language=bash, label=lst-bash-new-script-02, caption={Criando um script.}, postbreak=\mbox{$\hookrightarrow$\space}, basicstyle=\fontsize{8}{10}\selectfont\ttfamily]
#!/bin/bash
printf "Hello %s!\n" "$1"
\end{lstlisting}

\begin{lstlisting}[language=bash, label=lst-bash-convert-imgs, caption={Script para converter PNGs em JPGs}, postbreak=\mbox{$\hookrightarrow$\space}, basicstyle=\fontsize{8}{10}\selectfont\ttfamily]
for img in $( ls *.png ); do convert $img  ${img%%png}jpg; done
\end{lstlisting}

\begin{lstlisting}[language=bash, label=lst-bash-sequencia, caption={Sequência de 1 a 100.}, postbreak=\mbox{$\hookrightarrow$\space}, basicstyle=\fontsize{8}{10}\selectfont\ttfamily]
for i in `seq 100`; do echo $i; sleep 1; done
\end{lstlisting}

\begin{lstlisting}[language=bash, label=lst-bash-listpdfs, caption={Listar PDFs em ordem decrescente de tamanho}, postbreak=\mbox{$\hookrightarrow$\space}, basicstyle=\fontsize{8}{10}\selectfont\ttfamily]
ls -la *.[pP][dD][fF] | tr -s [:blank:] | cut -d' ' -f5,9 | column -t | sort -nr > pdflist.txt
\end{lstlisting}

Lista de alguns comandos uteis:
\begin{description}
\item[ls] listar conteúdo de diretórios
\item[echo] imprimir texto na tela
\item[touch] criar arquivo
\item[mkdir] criar diretório
\item[grep] procurar textos e padrões
\item[man] manual, obter ajuda para os comandos
\item[pwd] diretório corrente
\item[cd] trocar de diretório
\item[mv] mover arquivo
\item[rm] remover arquivo
\item[rmdir] remover diretório
\item[cp] copiar aquivos
\item[less] visualizar o conteúdo de arquivos
\item[cat] concatenar e imprimir na saída padrão
\item[>, <, |] redirecionamentos
\item[head] lê o início de um arquivo
\item[tail] lê o final de um arquivo
\item[chmod] alterar permissões
\item[kill] matar um processo
\item[sleep] esperar um determinado tempo
\item[\&] executar em background
\item[cal] calendário
\item[diff] verificar diferenças entre arquivos
\item[paste] juntar linhas de arquivos
\item[ps] status de processos
\item[sort] ordenar
\item[tr] traduzir, comprimir e/ou remover caracteres
\item[sed] editor de fluxo
\item[awk] linguagem para processamento de texto
\end{description}


\framebreak

\fullcite{evans2021}
\vspace{2ex}

\fullcite{devhints}
\vspace{2ex}

\fullcite{robbins2016}

\end{frame}


