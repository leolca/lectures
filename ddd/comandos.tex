\subsection{Comandos}

\begin{frame}[fragile,allowframebreaks]
\frametitle{Comandos}
Kunth definiu 325 primitivas para o \TeX{}.

\vspace{3ex}
O outros motores utilizam mais primitivas.
Veja: \hrefcolor{https://www.overleaf.com/learn/latex/TeX_primitives_listed_by_TeX_engine}{\TeX{} primitives listed by \TeX{ engine}}.


\vspace{3ex}
Outros comandos são definidos como combinações de primitivas ou de outros comandos.

\framebreak 

A formatação com \LaTeX{} é facilitada com a utilização de comandos.
Exemplos de comandos: \verb|\textbf{...}|, \verb|\url{...}|, \verb|\item|, etc.

\vspace{3ex}
Novos commandos podem ser definidos:
\begin{lstlisting}[language=tex, label=lst-comand-def, postbreak=\mbox{$\hookrightarrow$\space}, basicstyle=\fontsize{8}{10}\selectfont\ttfamily]
% comando simples (incluir \usepackage{amsfonts})
\newcommand{\R}{$\mathbb{R}$}

% comando com parametro
\newcommand{\bb}[1]{$\mathbb{#1}$} 

% (incluir \usepackage{hyperref})
\newcommand{\email}[1]{\href{mailto:#1}{#1}}
\end{lstlisting}

\framebreak

Comando com parâmetro opcional:
\begin{LTXexample}
\newcommand{\plusbinomial}[3][2]{(#2 + #3)^#1}

\[ \plusbinomial{x}{y} \]

\[ \plusbinomial[4]{y}{y} \]
\end{LTXexample}
\end{frame}



