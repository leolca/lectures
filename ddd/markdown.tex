\section{Markdown}

\begin{frame}[fragile,allowframebreaks]
\frametitle{Markdown}
*Markdown* é uma linguagem simples de marcação.

\vspace{3ex}
Pode ser utilizada para gerar documentos HTML, RTF, TeX, etc.

\vspace{3ex}
É utilizada (com algumas Variações) em sites como GitHub, Reddit, Diaspora, Stack Exchange, etc.

\vspace{3ex}
A Wikipedia também utiliza uma linguagem simples de marcação, chamada de \emph{wikitext} ou \emph{marcação wiki} ou \emph{wikicode}.

\framebreak 

\lstinputlisting[label=lst-ex-markdown, firstline=1, lastline=16, postbreak=\mbox{$\hookrightarrow$\space}, basicstyle=\fontsize{8}{10}\selectfont\ttfamily]{examples/markdown-ex1.txt}

\framebreak

\lstinputlisting[label=lst-ex-markdown, firstline=18, lastline=32, postbreak=\mbox{$\hookrightarrow$\space}, basicstyle=\fontsize{8}{10}\selectfont\ttfamily]{examples/markdown-ex1.txt}

\framebreak

\lstinputlisting[label=lst-ex-markdown, firstline=34, lastline=46, postbreak=\mbox{$\hookrightarrow$\space}, basicstyle=\fontsize{8}{10}\selectfont\ttfamily]{examples/markdown-ex1.txt}

\framebreak

\begin{itemize}
\item \hrefcolor{https://pandoc.org/}{pandoc} - conversor de documentos
\item \hrefcolor{https://www.ctan.org/pkg/markdown}{pacote de markdown para \LaTeX{}}
\end{itemize}

\end{frame}


