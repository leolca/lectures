%========================================================
% rodap\'e deve estar em espa\c{c}amento simples,
% independente do espa\c{c}amento no corpo de documento
% Devera ser feito com linespread{} \@currsize
% pois comandos do setspace nao funciona como desejado
%========================================================
% foi copiado do book e alterado
\makeatletter % ativa uso de ``@'' no nome
\renewcommand\@makefntext[1]{%
    \parindent 1em%
    % inicio da altera\c{c}\~ao    
    \linespread{1} \@currsize \noindent
    % \hb@xt@1.8em{\hss\@makefnmark}#1}
    \hb@xt@0.45em{\hss\@makefnmark}#1}
% fim da altera\c{c}\~ao
\makeatother % desativa o uso de ``@'' no nome

\let\orilstlistingname\lstlistingname
\renewcommand{\lstlistingname}{Exemplo}
\renewcommand{\lstlistlistingname}{Lista de \lstlistingname s}
% setup  listings
\lstset{
%  language=[LaTeX]TeX,
  numbers=none,
  backgroundcolor=\color{yellow!30},
  breaklines=true,
  breakautoindent=true,
  % breakatwhitespace=true,    % quebra linha somente no espaço
  belowcaptionskip=1\baselineskip, % espaço após o título
  % aboveskip=2ex,
  %keepspaces=true, % to keep indentation
  columns=fullflexible, % para poder copiar codigo do PDF (fixed, flexible, fullflexible)
  frame=lines, %none, single, line % do codigo
}

% aplicando patch no codigo de showexpl para que o parametro abovecaptionskip funcione tambem no LTXexample

\makeatletter
\renewcommand\SX@KillAboveCaptionskip{%
  \ifx\lst@caption\@empty\else
    \lst@IfSubstring t\lst@captionpos
      {\vskip-\abovecaptionskip}{}%
  \fi
  \ifx\lst@belowcaption\undefined \else %adicionado
    \belowcaptionskip=\lst@belowcaption\relax % adicionado
  \fi %adicionado
}
\makeatother

% setup showexpl
\lstset{explpreset={
  pos=b,
  belowcaptionskip=0.5\baselineskip, % espaço após o título
%  language=[LaTeX]TeX,
  %basicstyle=\ttfamily\small,
  %identifierstyle=\color{black},
  %keywordstyle=\color{blue},
  %commentstyle=\color{red},
  %extendedchars=true,
  % extendedchars=false
  %showspaces=false,
  %showstringspaces=false,
%  numbers=none,
  %numberstyle=\tiny,
%  breaklines=true,
  %backgroundcolor=\color{lightgray},
%  breakautoindent=true,
  %captionpos=b,
  %xleftmargin=0pt,
  % captionpos=t.
  % aboveskip=2ex,
  % columns=fullflexible, 
%  columns=flexible, % para poder copiar codigo do PDF
  % para editor (sem ele, terq espaço indesejado)
  % from http://latex.org/forum/viewtopic.php?t=5636
  % hsep=2ex,  
%  frame=lines, %none, single, line % do codigo
  %rframe={single} % do resultado
  }
}

% para arquivo de BibTeX no listings
% https://tex.stackexchange.com/questions/85998/include-bibtex-contents-in-a-listings-block 
\lstdefinelanguage{BibTeX}
  {keywords={%
      @article,@book,@collectedbook,@conference,@electronic,@ieeetranbstctl,%
      @inbook,@incollectedbook,@incollection,@injournal,@inproceedings,%
      @manual,@mastersthesis,@misc,@patent,@periodical,@phdthesis,@preamble,%
      @proceedings,@standard,@string,@techreport,@unpublished%
      },
   comment=[l][\itshape]{@comment},
   sensitive=false,
  }

\lstdefinestyle{lstexample}
{
%  language=[LaTeX]TeX,
%  keywordstyle=\bfseries,
%  identifierstyle=\itshape,
  backgroundcolor=\color{white},
  numbers=none,
  breaklines=true,
  breakautoindent=true,
%  belowcaptionskip=1\baselineskip, % espaço após o título
%  aboveskip=2ex,
  %keepspaces=true, % to keep indentation
  columns=fullflexible, % para poder copiar codigo do PDF
  frame=lines, %none, single, line % do codigo
%  explpreset={
%    attachfile=false,true
%    codefile=temporaryfilename,
%    explpreset=? defualti values is?
%    graphic=alternativegraphicsfileforoutput,
%    hsep=,
%    justification=\raggedright,% \raggedleft, \raggedright, \centering
%    overhang=,
%    pos=, % e t, b, l, r, and i
%    preset={},
%    rangeaccept=false, % true/false
%    rframe=,
%    varwidth=,
%    hsep,
%    wide,
%    width=
%  }
}

% LTXexample's parameter
%    attachfile=false,trure
%    codefile=temporaryfilename,
%    explpreset=? defualti values is?
%    graphic=alternativegraphicsfileforoutput,
%    hsep=,
%    justification=\raggedright,% \raggedleft, \raggedright, \centering
%    overhang=,
%    pos=, % e t, b, l, r, and i
%    preset={},
%    rangeaccept=false, % true/false
%    rframe=,
%    varwidth=,
%    hsep,
%    wide,
%    width=
%  }

% showexpl's LTXexample environment like ltxlisting
% is implemented (it use showexpl's preset (thus, need showexpl package}

\makeatletter 

% save some values for lstlinting
\newcounter{lstorisection}
\newcounter{lstorisubsection}
%\let\orilabel\label
%\let\oriref\ref
\newcommand{\emptyaddcontentsline}[3]{}
\let\oriaddcontentsline\addcontentsline

\newcounter{orilstlisting}

%\let\oriref\ref % without link
%\makeatletter
%\let\refstar\@refstar % without link
%\makeatother

% Example of untf8 verbatim write
%\newenvironment{writefile}[1]
% {\deactivateeightbit\VerbatimOut{#1.out}}
% {\endVerbatimOut}
%\makeatletter
%\newcommand{\deactivateeightbit}{%
%  \count@=127
%  \loop
%    \catcode\count@=12
%    \ifnum\count@<255
%    \advance\count@\@ne
%  \repeat
%}
%\makeatother

% from https://tex.stackexchange.com/questions/192941/showexpl-not-showing-caption-of-a-lstlisting-within to create vertical space
%\def\xstrut{\protect\rule[-2ex]{0pt}{2ex}}
%----------------------------------------------------------------

% showexpl's LTXexample preset
%\newcommand*\SX@@preset{%
%\renewcommand\documentclass[2][]{\SX@eat@version}%
%\renewcommand\usepackage[2][]{\SX@eat@version}%
%\renewenvironment{document}{}{}%
%\renewenvironment{figure}[1][]{\def\@captype{figure}}{}%
%\renewenvironment{table}[1][]{\def\@captype{table}}{}%
%\renewcommand\cite[1][]{}%
%\let\tableofcontens\relax \let\listoffigures\relax
%\let\listoftables\relax \let\printindex\relax
%\let\listfiles\relax \let\nofiles\relax
%\let\index\@gobble \let\label\@gobble
%\let\bibliography\@gobble
%\let\pagestyle\@gobble \let\thispagestyle\@gobble
%%%\let\immediate\relax \let\write\@gobbletwo
%%%\let\closeout\@gobble \let\@@input\@gobble
%\renewcommand\marginpar[2][]{}%
%\renewcommand\footnote[2][]{}%
%\let\@footnotetext\@gobble
%%%\abovedisplayskip=\z@
%%%\abovedisplayshortskip=\z@
%}
\makeatletter
\providecommand\SX@eat@version[1][]{}

\def\ltxlistingpresetfromshowexpl{%
\renewcommand\documentclass[2][]{\SX@eat@version}%
\renewcommand\usepackage[2][]{\SX@eat@version}%
\renewenvironment{document}{}{}%
\renewenvironment{figure}[1][]{\def\@captype{figure}}{}%
\renewenvironment{table}[1][]{\def\@captype{table}}{}%
\renewcommand\cite[1][]{}%
\let\tableofcontens\relax \let\listoffigures\relax
\let\listoftables\relax \let\printindex\relax
\let\listfiles\relax \let\nofiles\relax
\let\index\@gobble 
% \let\label\@gobble
\let\bibliography\@gobble
\let\pagestyle\@gobble \let\thispagestyle\@gobble
%%\let\immediate\relax \let\write\@gobbletwo
%%\let\closeout\@gobble \let\@@input\@gobble
\renewcommand\marginpar[2][]{}%
%\renewcommand\footnote[2][]{}%
%\let\@footnotetext\@gobble
%%\abovedisplayskip=\z@
%%\abovedisplayshortskip=\z@
} % showexpl default preset

\makeatother

% ltxlisting preset
\newcommand{\ltxlistingpreset}{
 \ltxlistingpresetfromshowexpl
 %
 \setcounter{lstorisection}{\value{section}}%
 \setcounter{section}{0}%
 \setcounter{lstorisubsection}{\value{subsection}}%
 \setcounter{subsection}{0}%
 \let\oriaddcontentsline\addcontentsline%
 \let\addcontentsline\emptyaddcontentsline%
 \let\orimathcal\mathcal%
 \let\orimathbb\mathbb%
 \let\mathcal\amsmathcal%
 \let\mathbb\amsmathbb%
 \let\orimaketitle\maketitle
 \let\maketitle\makearticletitle
 \let\title\orititle      
 \let\author\oriauthor
 \let\date\oridate
 \let\orithempfootnote\thempfootnote
 \renewcommand{\thempfootnote}{\arabic{mpfootnote}}
 \setcounter{orilstlisting}{\value{lstlisting}}
}

\newcommand{\ltxlistingposset}{
 \setcounter{section}{\value{lstorisection}}%
 \setcounter{subsection}{\value{lstorisubsection}}%
 \let\addcontentsline\oriaddcontentsline%
 \let\mathcal\orimathcal%
 \let\mathbb\orimathbb%
 \let\maketitle\orimaketitle
 \let\thempfootnote\orithempfootnote
 \setcounter{lstlisting}{\value{orilstlisting}}
}

\makeatletter
% from
% https://tex.stackexchange.com/questions/9035/how-to-pass-an-optional-argument-to-an-environment-with-verbatim-content
%
% \expandafter trick from
% https://tex.stackexchange.com/questions/196655/how-to-pass-figure-position-to-keyval-macro
\newenvironment{ltxlisting}{%
    \catcode`\^^M=\active
    \@ifnextchar[% se tem argumento opcional
        {\catcode`\^^M=5 \lxt@listing@start}
        {\catcode`\^^M=5 \lxt@listing@start[]}
}{\ltx@listing@end}

% Use of keyval from
% http://www.tex.ac.uk/FAQ-keyval.html
%
% Trick to ignore unkonown keys from
% https://tex.stackexchange.com/questions/314944/ignore-undefined-key-in-argument-its-defined-in-the-macro-before-its-used

\define@key{ltxlisting}{preset}{#1} % executte preset
% como eh dificil expandir segundo argumento antes do comando, define um comando auxiliar. Depois usar \expanafter para expandir argumento antes
\newcommand{\ltx@listing@setkeys}[1]{\setkeys{ltxlisting}{#1}}

%\def\lxt@listing@start[#1]{%
%  \begingroup
%    \ltxlistingpreset
%    % own preset
%    \let\KV@errx@ORI\KV@errx   % Save original error handling
%    \let\KV@errx\@gobble       % Ignore unknown keys
%      \setkeys{ltxlisting}{#1}
%    \let\KV@errx\KV@errx@ORI   % Restore original error handling      
%    %    
%    \def\ltx@listing@args{language={[LaTeX]TeX},#1}
%    \VerbatimEnvironment
%    % \deactivateeightbit
%    \begin{VerbatimOut}{\jobname.tmp}%
%}
%
%\def\ltx@listing@end{%
% \end{VerbatimOut}
% \begin{singlespacing}%
% \expandafter\lstinputlisting\expandafter[\ltx@listing@args]{\jobname.tmp}
% %\noindent
% \begin{framed} % frmae the output
% \noindent
% \begin{minipage}{\linewidth} % box the output
% \input{\jobname.tmp}
% \end{minipage}
% \end{framed}
% \end{singlespacing}%
% \ltxlistingposset
%\endgroup
%}

\def\lxt@listing@start[#1]{%
  %\begingroup
    \def\ltx@listing@args{language={[LaTeX]TeX},#1}
    \VerbatimEnvironment
    % \deactivateeightbit
    \begin{VerbatimOut}{\jobname.tmp}%
}

\def\ltx@listing@end{%
 \end{VerbatimOut}    %    
 \begin{singlespacing}
 \expandafter\lstinputlisting\expandafter[\ltx@listing@args]{\jobname.tmp}
 \end{singlespacing}
% \begingroup
% \begin{singlespacing}%
%   \ltxlistingpreset
%    % own preset
%   \let\KV@errx@ORI\KV@errx   % Save original error handling
%   \let\KV@errx\@gobble       % Ignore unknown keys
%     \expandafter\ltx@listing@setkeys\expandafter{\ltx@listing@args}
%   \let\KV@errx\KV@errx@ORI   % Restore original error handling      
% %\noindent
% \begin{framed} % frmae the output
% \noindent
% \begin{minipage}{\linewidth} % box the output
% \input{\jobname.tmp}
% \end{minipage}
% \end{framed}
% \end{singlespacing}%
% \ltxlistingposset
%\endgroup
%
\begin{ltxlistingout}[\ltx@listing@args]
\input{\jobname.tmp}
\end{ltxlistingout}
}

\newenvironment{ltxlistingout}[1][]
{ \begingroup
  \begin{singlespacing}%
  \ltxlistingpreset
  % own preset
  \let\KV@errx@ORI\KV@errx   % Save original error handling
  \let\KV@errx\@gobble       % Ignore unknown keys
    \expandafter\ltx@listing@setkeys\expandafter{#1}  
%    \expandafter\ltx@listing@setkeys\expandafter{\ltx@listing@args}
  \let\KV@errx\KV@errx@ORI   % Restore original error handling      
  %\noindent
  \begin{framed} % frame the output
  \noindent
  \begin{minipage}{\linewidth} % box the output
}{\end{minipage}
  \end{framed}
  \end{singlespacing}%
  \ltxlistingposset
  \endgroup
}

%%%%%%%%%%%%%%%%%%%%%%%%%%%%%%%%%%%%%%%%