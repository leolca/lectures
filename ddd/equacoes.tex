\subsection{Equações}
\begin{frame}[fragile]
\frametitle{Equações}
O \LaTeX{} contém as ferramentas necessárias para escrever equações em um documento simples.
Para um documento científico, deve-se utilizar os pacotes \texttt{amsmath} ou \texttt{mathtools}.

\begin{verbatim}
\usepackage{amsmath}
\end{verbatim}

Inserindo fórmulas:
  \begin{itemize}
  \item \emph{inline} (no meio do texto) utilize \verb|\( ... \)| ou \verb|$ ... $|
  \item para equações destacadas do texto utilize \verb|\[...\]| ou \verb|$$...$$| ou o ambiente \texttt{equation} ou \texttt{align}
  \end{itemize}
\end{frame}

\begin{frame}[fragile,allowframebreaks]
\frametitle{Equações}
\framesubtitle{Exemplos}

\begin{LTXexample}
\lipsum[1][1]
$\forall x \in X, \quad \exists y \leq \epsilon$
\lipsum[1][2]
\end{LTXexample}

\begin{LTXexample}
\(\alpha, \beta, \gamma, \delta, \epsilon, \zeta, \eta, \theta, \Gamma, \Delta, \Theta, \Lambda, \pi, \Pi, \phi, \Phi\)
\end{LTXexample}

\begin{LTXexample}
\lipsum[1][1]
\begin{equation}
\cos (2\theta) = \cos^2 \theta - \sin^2 \theta
\end{equation}
\lipsum[1][2]
\end{LTXexample}

\begin{LTXexample}
\lipsum[1][1-2]
\[ \lim_{x \to \infty} \exp(-x) = 0 \]
\lipsum[1][3-4]
\end{LTXexample}

\begin{LTXexample}
$x \equiv a \pmod b$
\end{LTXexample}

\begin{LTXexample}
$k_{n+1} = n^2 + k_n^2 - k_{n-1}$
\end{LTXexample}

\begin{LTXexample}
\begin{equation}
f(n) = \left. n^5 + 4n^2 + 2 \right|_{n=17}
\end{equation}
\end{LTXexample}

\begin{LTXexample}
$(\cdot), [\cdot], \{\cdot\}, |\cdot|, \lVert\cdot\rVert, \langle\cdot\rangle, \lfloor\cdot\rfloor, \lceil\cdot\rceil$
\end{LTXexample}

\begin{LTXexample}
\begin{equation}
\frac{n!}{k!(n-k)!} = \binom{n}{k} = {n \choose k}
\end{equation}
\end{LTXexample}

\begin{LTXexample}
\begin{equation}
\frac{\frac{1}{x}+\frac{1}{y}}{y-z}
\end{equation}
\end{LTXexample}

\begin{LTXexample}
\begin{equation}
x = a_0 + \frac{1}{a_1 + \frac{1}{a_2 + \frac{1}{a_3 + a_4}}}
\end{equation}
\end{LTXexample}

\begin{LTXexample}
\begin{equation}
\frac{
    \begin{array}[b]{r}
      \left( x_1 x_2 \right)\\
      \times \left( x'_1 x'_2 \right)
    \end{array}
  }{
    \left( y_1y_2y_3y_4 \right)
  }
\end{equation}
\end{LTXexample}

\begin{LTXexample}
\begin{equation}
\sqrt[n]{1+x+x^2+x^3+\ldots}
\end{equation}
\end{LTXexample}

\begin{LTXexample}
\lipsum[1][1] $\sum_{i=1}^{10} t_i$ \lipsum[1][2]
\end{LTXexample}

\begin{LTXexample}
\lipsum[1][1] $$ \int_0^\infty e^{-x}\,\mathrm{d}x $$ \lipsum[1][2]
\end{LTXexample}

\begin{LTXexample}
\begin{equation}
 \sum_{\substack{
    0<i<m \\
    0<j<n
 }}
 P(i,j)
\end{equation}
\end{LTXexample}

\begin{LTXexample}
$$\int\limits_a^b$$
\end{LTXexample}

\begin{LTXexample}
$\prod \bigoplus \bigotimes \bigcup \bigcap \oint \iint \iiint$
\end{LTXexample}

\begin{LTXexample}
$$\left(\frac{x^2}{y^3}\right)$$
\end{LTXexample}

\begin{LTXexample}
$$\left.\frac{x^3}{3}\right|_0^1$$
\end{LTXexample}

\begin{LTXexample}
\begin{equation}
\begin{matrix}
  a & b & c \\
  d & e & f \\
  g & h & i
 \end{matrix}
\end{equation}
\end{LTXexample}

\begin{LTXexample}
\begin{equation}
\label{eqn-Amn}
 A_{m,n} =
 \begin{pmatrix}
  a_{1,1} & a_{1,2} & \cdots & a_{1,n} \\
  a_{2,1} & a_{2,2} & \cdots & a_{2,n} \\
  \vdots  & \vdots  & \ddots & \vdots  \\
  a_{m,1} & a_{m,2} & \cdots & a_{m,n}
 \end{pmatrix}
\end{equation}

Conforme a Eq. \ref{eqn-Amn}.
\end{LTXexample}

\begin{LTXexample}
\begin{equation}
f(n) = \left\{ 
\begin{array}{l l}
n/2 & \quad \text{if $n$ is even}\\
-(n+1)/2 & \quad \text{if $n$ is odd}\\
\end{array} \right.
\end{equation}
\end{LTXexample}

\begin{LTXexample}
\begin{eqnarray*}
\cos 2\theta & = & \cos^2 \theta - 
                   \sin^2 \theta \\
             & = & 2 \cos^2 \theta - 1.
\end{eqnarray*}
\end{LTXexample}

\begin{LTXexample}
\begin{align*}
  z_0 &= d = 0 \\
  z_{n+1} &= z_n^2+c
\end{align*}
\end{LTXexample}

\end{frame}


\begin{frame}
\frametitle{Mais informações e exemplos}

\begin{itemize}
\item \href{http://tug.ctan.org/info/short-math-guide/short-math-guide.pdf}{Short Math Guide for \LaTeX}
\item \url{https://www.overleaf.com/learn/latex/Mathematical_expressions}
\item \url{https://en.wikibooks.org/wiki/LaTeX/Mathematics}
\item \url{https://en.wikibooks.org/wiki/LaTeX/Advanced_Mathematics}
\item \url{https://en.wikibooks.org/wiki/LaTeX/Theorems}
\end{itemize}

\end{frame}


\begin{frame}
\frametitle{Dicas para iniciantes}

\begin{itemize}
\item \hrefcolor{https://detexify.kirelabs.org}{Detexify}
\item \hrefcolor{https://www.latex4technics.com/}{LaTeX4technics}
\item \hrefcolor{https://www.mathcha.io/}{Editor de equações online}
\item \hrefcolor{http://www.wolframalpha.com/input/?i=int+sin{x^2}\%2Bsqrt{x}+dx}{Notação TeX e computação no Wolfram Alpha}
\end{itemize}
\end{frame}
