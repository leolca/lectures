\subsection{Códigos}

\begin{frame}%[allowframebreaks]
\frametitle{Códigos}
\LaTeX{} é muito utilizando nas áreas de ciências e computação, sendo assim importante
ter ferramentas adequadas para documentação de códigos fonte.

\begin{itemize}
\item ambiente \texttt{verbatim}\footnote{Verbatim: literalmente, ipsis litteris, idêntico à fonte original, palavra por palavra.}
\item pacote \texttt{listings}: \url{https://www.ctan.org/pkg/listings}
\item pacote \texttt{minted}: \url{https://www.ctan.org/pkg/minted}
\end{itemize}

\end{frame}


\begin{frame}[fragile]%[allowframebreaks]
\frametitle{Verbatim}

\begin{lstlisting}[language=tex, label=lst-verbatim, caption={Uso do ambiente verbatim.}, postbreak=\mbox{$\hookrightarrow$\space}, basicstyle=\fontsize{8}{10}\selectfont\ttfamily]
\begin{verbatim}
Exemplo \textbf{verbatim}.
$ x = a^2 + 2bc $
\end{verbatim}
\end{lstlisting}

Resultado da Lista \ref{lst-verbatim}:
\begin{verbatim}
Exemplo \textbf{verbatim}.
$ x = a^2 + 2bc $
\end{verbatim}

\end{frame}


\begin{frame}[fragile,allowframebreaks]
\frametitle{listings}

\begin{footnotesize}
\begin{lstlisting}[language=Python, label=lst-listings1, caption={Inserindo o código direto no aquiro \texttt{.tex}.}, postbreak=\mbox{$\hookrightarrow$\space}, basicstyle=\fontsize{8}{10}\selectfont\ttfamily]
def factorial(n):
    if n>1:
        F=n*factorial(n-1)
    else:
        F=1
    return F
\end{lstlisting}

\begin{verbatim}
\begin{lstlisting}[language=Python]
def factorial(n):
    if n>1:
        F=n*factorial(n-1)
    else:
        F=1
    return F
\end{lstlisting}
\end{verbatim}
\end{footnotesize}

\framebreak

Importando o código de um arquivo.
\begin{verbatim}
\lstinputlisting[language=Python]{factorial.py}
\end{verbatim}
\lstinputlisting[language=Python]{factorial.py}

\framebreak
\begin{verbatim}
\lstinputlisting[language=Python,firstline=2, lastline=5]{factorial.py}
\end{verbatim}
\lstinputlisting[language=Python,firstline=2, lastline=5]{factorial.py}


\framebreak
Suporte a diversas linguagens: bash, C, C++, Java, HTML, Matlab, Octave, Perl, Python, R, TeX, XML, etc.
\vspace{3ex}

Opções de customização: \\
backgroundcolor, 
commentstyle,
basicstyle (e.g. \verb|basicstyle=\ttfamily\small|),
keywordstyle (e.g. \verb|keywordstyle=\color{red}|),
numberstyle,
numbersep,
stringstyle,
showspaces,
showstringspaces,
showtabs,
numbers, 
prebreak,
captionpos,
frame,
breakwhitespace,
breaklines,
keepspaces,
tabsize,
escapeinside,
rulecolor, etc.

\end{frame}

